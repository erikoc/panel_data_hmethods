\documentclass[12pt]{article}
\usepackage[a4paper,lmargin=3cm, rmargin = 2cm, tmargin = 2cm, bmargin = 2cm ]{geometry}

\usepackage{mathptmx}
\usepackage{graphicx} % Required for inserting images
\usepackage{setspace}
\renewcommand{\baselinestretch}{1.5} 
\usepackage{enumitem}
\usepackage{amsmath,amsthm,amssymb,amsfonts}
\usepackage{wrapfig}
\usepackage{lmodern,mathrsfs}
\usepackage{tikz,tikz-3dplot,tikz-cd,tkz-tab,tkz-euclide,pgf,pgfplots}
\usepackage{hyperref}
\usepackage[nameinlink]{cleveref}
\usepackage{natbib}
\usepackage{multirow}
\usepackage{multicol}
\usepackage[table]{xcolor}
\usepackage{tabularx}
\usepackage{rotating}
\usepackage{booktabs}
\usepackage{blindtext}
\usepackage{comment}
\usepackage[title]{appendix}
\usepackage{fancyhdr}
\usepackage[justification=centering]{caption}
\bibliographystyle{apa-good}
\newcommand{\ep}{\varepsilon}
\newcommand{\vp}{\varphi}
\newcommand{\lam}{\lambda}
\newcommand{\Lam}{\Lambda}
%\newcommand{\abs}[1]{\ensuremath{\left\lvert#1\right\rvert}} % This clashes with the physics package
%\newcommand{\norm}[1]{\ensuremath{\left\lVert#1\right\rVert}} % This clashes with the physics package
\renewcommand{\ip}[1]{\ensuremath{\left\langle#1\right\rangle}}
\newcommand{\floor}[1]{\ensuremath{\left\lfloor#1\right\rfloor}}
\newcommand{\ceil}[1]{\ensuremath{\left\lceil#1\right\rceil}}
\newcommand{\A}{\mathbb{A}}
\newcommand{\B}{\mathbb{B}}
\newcommand{\C}{\mathbb{C}}
\newcommand{\D}{\mathbb{D}}
\newcommand{\E}{\mathbb{E}}
\newcommand{\F}{\mathbb{F}}
\newcommand{\K}{\mathbb{K}}
\newcommand{\N}{\mathbb{N}}
\newcommand{\Q}{\mathbb{Q}}
\newcommand{\R}{\mathbb{R}}
\newcommand{\T}{\mathbb{T}}
\newcommand{\X}{\mathbb{X}}
\newcommand{\Y}{\mathbb{Y}}
\newcommand{\Z}{\mathbb{Z}}
\newcommand{\As}{\mathcal{A}}
\newcommand{\Bs}{\mathcal{B}}
\newcommand{\Cs}{\mathcal{C}}
\newcommand{\Ds}{\mathcal{D}}
\newcommand{\Es}{\mathcal{E}}
\newcommand{\Fs}{\mathcal{F}}
\newcommand{\Gs}{\mathcal{G}}
\newcommand{\Hs}{\mathcal{H}}
\newcommand{\Is}{\mathcal{I}}
\newcommand{\Js}{\mathcal{J}}
\newcommand{\Ks}{\mathcal{K}}
\newcommand{\Ls}{\mathcal{L}}
\newcommand{\Ms}{\mathcal{M}}
\newcommand{\Ns}{\mathcal{N}}
\newcommand{\Os}{\mathcal{O}}
\newcommand{\Ps}{\mathcal{P}}
\newcommand{\Qs}{\mathcal{Q}}
\newcommand{\Rs}{\mathcal{R}}
\newcommand{\Ss}{\mathcal{S}}
\newcommand{\Ts}{\mathcal{T}}
\newcommand{\Us}{\mathcal{U}}
\newcommand{\Vs}{\mathcal{V}}
\newcommand{\Ws}{\mathcal{W}}
\newcommand{\Xs}{\mathcal{X}}
\newcommand{\Ys}{\mathcal{Y}}
\newcommand{\Zs}{\mathcal{Z}}
\newcommand{\ab}{\textbf{a}}
\newcommand{\bb}{\textbf{b}}
\newcommand{\cb}{\textbf{c}}
\newcommand{\db}{\textbf{d}}
\newcommand{\ub}{\textbf{u}}
%\renewcommand{\vb}{\textbf{v}} % This clashes with the physics package (the physics package already defines the \vb command)
\newcommand{\wb}{\textbf{w}}
\newcommand{\xb}{\textbf{x}}
\newcommand{\yb}{\textbf{y}}
\newcommand{\zb}{\textbf{z}}
\newcommand{\Ab}{\textbf{A}}
\newcommand{\Bb}{\textbf{B}}
\newcommand{\Cb}{\textbf{C}}
\newcommand{\Db}{\textbf{D}}
\newcommand{\eb}{\textbf{e}}
\newcommand{\ex}{\textbf{e}_x}
\newcommand{\ey}{\textbf{e}_y}
\newcommand{\ez}{\textbf{e}_z}
\newcommand{\abar}{\overline{a}}
\newcommand{\bbar}{\overline{b}}
\newcommand{\cbar}{\overline{c}}
\newcommand{\dbar}{\overline{d}}
\newcommand{\ubar}{\overline{u}}
\newcommand{\vbar}{\overline{v}}
\newcommand{\wbar}{\overline{w}}
\newcommand{\xbar}{\overline{x}}
\newcommand{\ybar}{\overline{y}}
\newcommand{\zbar}{\overline{z}}
\newcommand{\Abar}{\overline{A}}
\newcommand{\Bbar}{\overline{B}}
\newcommand{\Cbar}{\overline{C}}
\newcommand{\Dbar}{\overline{D}}
\newcommand{\Ubar}{\overline{U}}
\newcommand{\Vbar}{\overline{V}}
\newcommand{\Wbar}{\overline{W}}
\newcommand{\Xbar}{\overline{X}}
\newcommand{\Ybar}{\overline{Y}}
\newcommand{\Zbar}{\overline{Z}}
\newcommand{\Aint}{A^\circ}
\newcommand{\Bint}{B^\circ}
\newcommand{\limk}{\lim_{k\to\infty}}
\newcommand{\limm}{\lim_{m\to\infty}}
\newcommand{\limn}{\lim_{n\to\infty}}
\newcommand{\limx}[1][a]{\lim_{x\to#1}}
\newcommand{\liminfm}{\liminf_{m\to\infty}}
\newcommand{\limsupm}{\limsup_{m\to\infty}}
\newcommand{\liminfn}{\liminf_{n\to\infty}}
\newcommand{\limsupn}{\limsup_{n\to\infty}}
\newcommand{\sumkn}{\sum_{k=1}^n}
\newcommand{\sumk}[1][1]{\sum_{k=#1}^\infty}
\newcommand{\summ}[1][1]{\sum_{m=#1}^\infty}
\newcommand{\sumn}[1][1]{\sum_{n=#1}^\infty}
\newcommand{\emp}{\varnothing}
\newcommand{\exc}{\backslash}
\newcommand{\sub}{\subseteq}
\newcommand{\sups}{\supseteq}
\newcommand{\capp}{\bigcap}
\newcommand{\cupp}{\bigcup}
\newcommand{\kupp}{\bigsqcup}
\newcommand{\cappkn}{\bigcap_{k=1}^n}
\newcommand{\cuppkn}{\bigcup_{k=1}^n}
\newcommand{\kuppkn}{\bigsqcup_{k=1}^n}
\newcommand{\cappk}[1][1]{\bigcap_{k=#1}^\infty}
\newcommand{\cuppk}[1][1]{\bigcup_{k=#1}^\infty}
\newcommand{\cappm}[1][1]{\bigcap_{m=#1}^\infty}
\newcommand{\cuppm}[1][1]{\bigcup_{m=#1}^\infty}
\newcommand{\cappn}[1][1]{\bigcap_{n=#1}^\infty}
\newcommand{\cuppn}[1][1]{\bigcup_{n=#1}^\infty}
\newcommand{\kuppk}[1][1]{\bigsqcup_{k=#1}^\infty}
\newcommand{\kuppm}[1][1]{\bigsqcup_{m=#1}^\infty}
\newcommand{\kuppn}[1][1]{\bigsqcup_{n=#1}^\infty}
\newcommand{\cappa}{\bigcap_{\alpha\in I}}
\newcommand{\cuppa}{\bigcup_{\alpha\in I}}
\newcommand{\kuppa}{\bigsqcup_{\alpha\in I}}
\newcommand{\Rx}{\overline{\mathbb{R}}}
\newcommand{\dx}{\,dx}
\newcommand{\dy}{\,dy}
\newcommand{\dt}{\,dt}
\newcommand{\dax}{\,d\alpha(x)}
\newcommand{\dbx}{\,d\beta(x)}
\DeclareMathOperator{\glb}{\text{glb}}
\DeclareMathOperator{\lub}{\text{lub}}
\newcommand{\xh}{\widehat{x}}
\newcommand{\yh}{\widehat{y}}
\newcommand{\zh}{\widehat{z}}
\newcommand{\<}{\langle}
\renewcommand{\>}{\rangle}
\renewcommand{\iff}{\Leftrightarrow}
\DeclareMathOperator{\im}{\text{im}}
\let\spn\relax\let\Re\relax\let\Im\relax
\DeclareMathOperator{\spn}{\text{span}}
\DeclareMathOperator{\Re}{\text{Re}}
\DeclareMathOperator{\Im}{\text{Im}}
\DeclareMathOperator{\diag}{\text{diag}}

\usepackage{acronym}
\acrodef{cdf}[cdf]{Cumulative Distribution Function}
\acrodef{pdf}[pdf]{Probability Density Function}
\acrodef{pmf}[pmf]{Probability Mass Function}
\acrodef{iid}[i.i.d.]{independent and identically distributed}
\acrodef{did}[DiD]{Difference-in-Difference}
\acrodef{ols}[OLS]{Ordinary Least Squares}
\acrodef{twfe}[TWFE]{Two-Way Fixed Effects}
\acrodef{twfer}[TWFEr]{Two-Way Fixed Effects Regression}
\acrodef{OWFE}[OWFE]{One-Way Fixed Effects}
\acrodef{TOST}[TOST]{Two One-Sided Tests}
\acrodef{GDP}[GDP]{Gross Domestic Product}
\acrodef{DGPs}[DGPs]{Data-Generating Processes}
\acrodef{DGP}[DGP]{Data-Generating Process}
\acrodef{KSS}[KSS]{Kneip, Sickles and Song}
\acrodef{Eup}[Eup]{Entirely Updated Estimators}
\acrodef{CV}[CV]{Cross-Validation}
\acrodef{GCV}[GCV]{Generalized Cross-Validation}
\acrodef{CI}[CI]{Confidence Interval}
\acrodef{MSE}[MSE]{Mean Squared Error}



\begin{document}

\setlength\tabcolsep{3.5pt}

\pagenumbering{arabic}

\begin{titlepage}
\begin{center}

\vspace*{1cm}

{\Large \textbf{Factor-Based Modeling of Fixed Effects in Panel Data Methods: A Comparative Analysis}}

\vspace{3cm}
Master Thesis Presented to the 

Department of Economics at the

Rheinische Friedrich-Wilhelms-Universität Bonn


\vspace{1cm}
In Partial Fulfillment of the Requirements for the Degree of

Master of Science (M.Sc.)

\vspace{6cm}

Supervisor: Prof. Dr. Dominik Liebl

\vfill
Submitted in August 2023 by:

Erik Ortiz Covarrubias 

Matriculation Number: 3395844

\end{titlepage}
\end{center}





\pagenumbering{gobble}
\tableofcontents
\clearpage

\pagenumbering{arabic}
\section{Introduction}

Panel data models offer the ability to manage unobserved endogeneity and allow for broader forms of heterogeneity \citep{hansen2022econometrics}. 

\citet{kneip2012new} propose estimating the time-varying individual effects nonparametrically. This is achieved through a two-stage procedure involving spline smoothing and principal component analysis. 

\section{The Model}

We assume balanced panel data with $n$ cross-sectional units and $T$ time periods. We aim to model the variation of an independent variable $y_{it}$ for  $i \in \{1, ..., n\}$ and $t \in \{1, ..., T\}$ in dependent explanatory variables $x_{it} \in \R^P$.  We consider the following panel data model:


\begin{equation}\label{model1}
    y_{it} = \sum_{j=1}^P x_{itj}\beta_j + \nu_{it} + \varepsilon_{it}
\end{equation}

Where $x_{itj}$ is the $jth$ element of the vector of independent variables, $\varepsilon_{it}$ are idiosyncratic errors, and $\nu_{it} \in \R $ are unobserved non-constant individual effects. Note that whenever $x_{it}$ includes an intercept, identifiability requires $v_{it}$ to be centred around zero. Otherwise, the non-constant individual effects are centred around the overall mean. 

Our main goal is to estimate and analyze  $v_{it}$. However, the estimation of $\beta$ remains of interest to us. We assume that $\nu_{it}$ has a factor structure which can be parametrized in terms of $d$ common factors as follows:

\begin{equation}\label{factors}
 \nu_{i t}=\left\{\begin{array}{c}v_{i t}=\sum_{l=1}^d \lambda_{i l} f_{l t} \\ v_i(t)=\sum_{l=1}^d \lambda_{i l} f_l(t)\end{array}\right.
\end{equation}


Here $\lambda_{il}$ are the individual loading parameters, $f_{lt}$ are the common factors of the general model of \citet{bai2009panel},  and $f_{l}(t)$ are the common factors for the model of \citet{kneip2012new}. %The type of models represented by equations (\ref{model1}) and (\ref{factors}) are known as interactive fixed-effects models. 

\citet{bai2009panel} treats both $\lambda_{il}$ and $f_{lt}$ as fixed-effects parameters to be jointly estimated with $\beta$. Heteroskedasticity and dependency across time and cross-sectional units are allowed. Further, $f_{lt}$ is modelled as an integrated or stationary process, which is allowed to have a non-zero mean.  Both the individual loading parameters and the common factors parameters are allowed to be correlated with the regressors $x_{it}$.

In contrast,  \citet{kneip2012new} model the time-varying individual effects as linear combinations of a small number of unknown basis functions ($f_l(t))$, where the individual loading parameters set the weight of every basis function for each cross-sectional unit. This translates into smooth, slowly varying local trends. The author's setup consequently allows for strongly correlated stationary and non-stationary factors. %not clear. 

This work will centre around the discussion of the model and estimation methods proposed by \citet{kneip2012new}, and its comparison to the general model synthesized by \citet{bai2009panel} and some of the estimation methods applicable to that setup. 

\subsection{Interactive and additive fixed-effects}

The specification in (\ref{model1})  includes the classic panel data models with additive fixed-effects model as a special case. Indeed, for $d=2$, a first common factor $f_{1t} = 1, \forall t$ with individual loading parameters $\lambda_{i1}$ and a second common factor of the form $f_{2t}$ with identical loading parameters $\lambda_{i2} = 1,  \forall i$ we get the classical two-way error component model:

\begin{equation*}
      y_{it} = \sum_{j=1}^\rho x_{itj}\beta_j + \lambda_{i1} + f_{2t} + \varepsilon_{it}
\end{equation*}

Nonetheless, unlike the case of classical additive effects panel models, well-known estimation methods, such as the within-transformation, are generally inadequate. To see this, consider the case where $d =1$, $ y_{it} = \sum_{j=1}^\rho x_{itj}\beta_j + \lambda_{i1}f_{1t} + \varepsilon_{it}$. Then, the within-transformation $ \dot{y}_{it}   =   y_{it} - \bar{y}_i = \sum_{j=1}^\rho (x_{it} - \bar{x}_i)\beta_j + \lambda_i(f_{1t} - \bar{f}) + \varepsilon_{it} - \bar{\varepsilon}_{i}$ is unable to eliminate the interactive effects since, generally, $f_{1t} \neq \bar{f}$. Hence, the within estimator is inconsistent for the model as the potential endogeneity between regressors and unobservables can't be addressed \citep{bai2009panel}. 

It is worth noting that since interactive-effects models encompass additive-effects models as a specific case, a consistent estimator for the former will also be consistent for the latter, albeit less efficient than the classical estimator \citep{bai2009panel}. To prevent inefficiency, it is possible to enhance model (\ref{model1}) by explicitly incorporating the classical additive effects,

\begin{equation}\label{explicit_model}
    y_{it} = \sum_{j=1}^\rho x_{itj}\beta_j + \nu_{it} + \alpha_i + \xi_t + \varepsilon_{it},
\end{equation}

where $\alpha_i$ and $\xi_t$ denote the unit- and time-specific fixed effects. This specification also allows for further interpretability. Model (\ref{explicit_model}) can be estimated via augmented versions of the methods proposed by both \citet{bai2009panel} and \citet{kneip2012new}. This is discussed in detail \citet{bai2009panel} and \citet{bada2012phtt}.

\subsection{Factor modelling in economics}

Model (\ref{model1}) can fit a wide range of economic phenomena where unobservables might be present as common factors. A few examples might elucidate the usefulness of the approach discussed in this paper.  

\paragraph{Macroeconomics} Let $y_{it}$ be the growth rate for a country $i$ in period $t$ and $x_{it}$ be a series of inputs, such as labour and capital. The factors $f_{lt}$ could represent common macroeconomic shocks such as technological change and financial crises, and the individual loading parameters $\lambda_{il}$ the heterogeneous impacts of such shocks to the countries' growth rates. 

\paragraph{Microeconomics} Consider a setup where $y_{it}$ represents the wage for individual.

\section{Identification}\label{identification}

In the absence of any further constraints, the problem of the non-uniqueness of common factors leads to indeterminacy. This, and the ensuing normalizations, are easier understood when using matrix notation. Let $Y_i = (y_{i1}, \ldots, y_{iT})^\prime$, $X_i = (x_{i1}, \ldots, x_{iT})$, $F = (F_1,  \ldots, F_T)^\prime$,$ \lambda_i = (\lambda_{i 1}, \ldots, \lambda_{i d})^\prime$ and $\varepsilon_{i} = (\varepsilon_{i1}, \ldots, \varepsilon_{iT})^\prime$, with  $F_t = (f_1(t), \ldots, f_d(t))^\prime$. We then write the model as:

\begin{equation}\label{matrix_notation}
    Y_i=X_i \beta+ F\lambda_i^\prime +\varepsilon_i
\end{equation}

 

Now, since, for any invertible  $d \times d$ matrix $A$, it holds that $ F A A^{-1} \lambda_i^\prime  =  F \lambda_i^\prime$, the model, with factors $FA$ and loading parameters $\lambda_i (A^{-1})^\prime$ is also true. Hence, factors are only identifiable up to linear transformations of the form presented above. Sine matrix $A$ has $d \times d$ free elements, we require $d^2$ restrictions on the model for identifiability. 


Consider $\Lambda = (\lambda_1, \ldots, \lambda_n)^\prime$. The usual normalizations are then given by:   

\begin{enumerate}[label = (\alph*)]
    \item $F^\prime F/T = I_d$ \label{cond_a}
    \item $\Lambda^\prime \Lambda = \diag(\sum_{i=1}^n \lambda_{i1}^2, \ldots, \sum_{i=1}^n \lambda_{id}^2 ) $ \label{cond_b}
\end{enumerate}



Where the former yields $\frac{d(d+1)}{2}$ restrictions, and the latter provides the additional $\frac{d(d-1)}{2}$ restrictions. Conditions \ref{cond_a} and \ref{cond_b} ensure identifiability up to sign changes since, e.g. $-F$ and $-\lambda$ also satisfy these restrictions \citep{bai2009panel, kneip2012new}.

The above normalizations lead to orthogonal vectors $F_t$ and empirically uncorrelated coefficients  $\lambda_{il}$. Further, under these restrictions, the problem of estimating factors $F_t$ becomes that of principal component analysis  \citep{kneip2012new}.


\section{Estimation}

\subsection{Method by \citet{kneip2012new}}
The estimation approach developed by \citet{kneip2012new} involves a two-step procedure. First, estimates for the common slopes  $\hat{\beta}_j$ and initial estimates for the time-varying individual effects $\tilde{v}_i(t)$ are obtained via least squares, where a roughness penalty $\kappa$ controls the smoothness of the latter. This first step of the estimation relies on the use of an auxiliary function $\vartheta_i(t)$ defined on the interval $[1,T]$, so that 
$\hat{\vartheta}_i(t) := \tilde{v}_i(t)$.

Second, principal component analysis is used to estimate the common factors $f_l(t)$ and produce a final and more efficient estimate $\hat{v}_i(t)$ for the non-constant individual effects. 

In what follows, each step will be discussed in detail. 

\paragraph{Step 1:} For a given $\kappa >0 $, the unobserved paramters $\beta_j$ and $v_i(t)$ are estimated by the minimization of


\begin{equation}\label{obj_kneip1}
    \sum_{i=1}^n \frac{1}{T} \sum_{t=1}^T\left(y_{i t}-\sum_{j=1}^P x_{i t j} \beta_j-\vartheta_i(t)\right)^2+\sum_{i=1}^n \kappa \int_1^T \frac{1}{T}\left(\vartheta_i^{(m)}(s)\right)^2 d s
\end{equation}

over all $\beta_j \in \R$ and all functions $\vartheta_i(t)$ of class $C^m$, where $\vartheta^{(m)}_i(t)$ denotes the $m$th derivative of $\vartheta_i(t)$. Spline theory implies that any solution $\hat{\vartheta}_i(t)$ has an expansion in terms of a natural spline basis $z_1(t), \ldots, z_T(t)$ of order $2m$ such that $\hat{\vartheta}_i(t) = \sum_{s=1}^T \hat{\zeta}_{is} z_s(t)$. For a treatment of spline theory and spline smoothing see e.g. \citet{eubank1999nonparametric}, \citet{wasserman2006all}.

Using the model in matrix notation (\ref{matrix_notation}) and the expansion of the time-varying individual effects, we can rewrite the objective function in  (\ref{obj_kneip1}) as:

\begin{equation}\label{matrix_objective}
    S(\beta, \zeta)=\sum_{i=1}^n\left(\left\|Y_i-X_i \beta-Z \zeta_i\right\|^2+\kappa \zeta_i^{\prime} R \zeta_i\right)
\end{equation}

here $\zeta_i = (\zeta_{i1}, \ldots, \zeta_{iT})^\prime$, $Z$ and $R$ are $T \times T$ matrices with elements $\{z_s(t)\}_{s,t = 1, \ldots, T}$ and $\{\int z_s^{(m)}(t)z_k^{(m)}(t), dt\}_{s,k=1, \ldots, T}$ respectively. Further, $\| \cdot\|$ denotes the eculidean norm in $\R^T$.   Estimators $\hat{\beta}, \hat{\zeta}_i$ and $\tilde{v}_i$ are hence obtained by minimizing (\ref{matrix_objective}) over all $\beta \in \R^\rho$ and $\zeta \in \R^{T \times n}$. With  $\mathcal{Z}_\kappa=Z\left(Z^{\prime} Z+\kappa R\right)^{-1} Z^{\prime}$, the solutions are given by: 

\begin{equation}
    \begin{aligned}
& \hat{\beta}=\left(\sum_{i=1}^n X_i^{\prime}\left(I-\mathcal{Z}_\kappa\right) X_i\right)^{-1}\left(\sum_{i=1}^n X_i^{\prime}\left(I-\mathcal{Z}_\kappa\right) Y_i\right) \\
& \hat{\zeta}_i=\left(Z^{\prime} Z+\kappa R\right)^{-1} Z^{\prime}\left(Y_i-X_i \hat{\beta}\right), \text { and } \\
& \tilde{v}_i=\mathcal{Z}_\kappa\left(Y_i-X_i \hat{\beta}\right),
\end{aligned}
\end{equation}

\paragraph{Step 2:}\label{step2} The common factors are obtained as the principal components of the sample $\tilde{v}_1, \ldots, \tilde{v}_n$. More precisely, let  

\begin{equation}\label{empirical_covar}
    \hat{\Sigma} = \frac{1}{n} \sum_{i=1}^n \tilde{v}_i \tilde{v}_i^\prime
\end{equation}

denote the empirical covariance matrix of $\tilde{v}_1, \ldots, \tilde{v}_n$. Let $\hat{\rho}_1 \geq,  \ldots, \geq \hat{\rho}_T$ and $\hat{\gamma}_1, \dots, \hat{\gamma}_T$ denote the eigenvalues and corresponding eigenvectors of (\ref{empirical_covar}). Then, the estimator of the common factor $f_l(t)$ is given by the $l$th scaled eigenvector

\begin{equation}\label{hat_f}
    \hat{f}_l(t) = \sqrt{T} \hat{\gamma}_{lt}, \, \text{for all} \, l = \{1, \ldots, d\}, t = \{t = 1, \ldots T\}
\end{equation}

where $\hat{\gamma}_{lt}$ is the $t$th element of the eigenvector $\hat{\gamma}_l$. The scaling factor $\sqrt{T}$ yields that (\ref{hat_f}) fullfils condition \ref{cond_a} in \Cref{identification}. The estimates $\hat{\lambda}_{il}$ for the individual loading paramters are obtianed via least squares of $(Y_i - X_i \hat{\beta})$ on $\hat{f}_l = (\hat{f}_l(1), \ldots, \hat{f}_l(T))^\prime$.

A crucial part of the method proposed by \citet{kneip2012new} involves re-estimating the time-varying individual effects $v_i(t)$ in Step 2 by $\hat{v}_i(t) := \sum_{l=1}^d \hat{\lambda}_{il}  \hat{f}_l(t) $, where the factor dimension $d$ is determined by a  stuiable dimensionality criterion. 
6
\subsection{Method by \citet{bai2009panel}}
\citet{bai2009panel} proposes to estimate parameters $\beta, F$ and $\lambda_i$ in model (\ref{matrix_notation}), given a known factor dimension $d$,  by minimizing the least squares objective function defined as:

\begin{equation}
    S\left(\beta, F, \lambda_i\right)=\sum_{i=1}^n\left\|Y_i-X_i \beta-F \lambda_i^{\prime}\right\|^2
\end{equation}

subject to conditions \ref{cond_a} and $\ref{cond_b}$ in \Cref{identification}. Given the projection matrix $\mathcal{P}_d = I_T - F(F^\prime F)^{-1}F^\prime = I_T - FF^\prime/T$, the least square estimator for $\beta$, given $F$ is given by:

\begin{equation}
    \hat{\beta}(F)=\left(\sum_{i=1}^n X_i^{\prime} \mathcal{P}_d X_i\right)^{-1}\left(\sum_{i=1}^n X_i^{\prime} \mathcal{P}_d Y_i\right)
\end{equation}

Now, given $\beta$, an estimator for $F$ is given, once again by the first $d$ eigenvectors $\hat{\gamma} = (\hat{\gamma}_1, \ldots, \hat{\gamma}_d)$  of the empirical covariance matrix $\hat{\Sigma} = (nT)^{-1}w_i w_i^\prime$, where $w_i = Y_i - X_i \beta$. More concretely:

\begin{equation}
    \hat{F}(\beta) = \sqrt{T}\hat{\gamma}
\end{equation}

The solution for $\hat{\beta}$ and $\hat{F}$ can then be obtained via iteration. Estimates for the individual loading parameters can then be obtained as follows:

\begin{equation}
    \hat{\lambda_i} = \frac{1}{T} \hat{F}^\prime (Y_i - X_i \hat{\beta}) 
\end{equation}


\citet{bada2014parameter} propose agumenting the method by \citet{bai2009panel} 



\section{Asymptotics}

\newtheorem{assumption}{Assumption}

\begin{assumption}
    For some fixed $d \in \{0,1,2,  \ldots\}, d < T$ there exists an $D$- dimensional subspace $\mathcal{D}_T$ of $\R^T$ such that $v_i \in \mathcal{D}_T$ holds with probability 1. 
\end{assumption}

\begin{assumption}
    There exist a nondecreasing function $c(T)$ of $T$ such that for all $l,k = 1, \ldots, d, l \neq k$:
        \begin{enumerate}
            \item $E( \frac{1}{T} \sum_{t=1}^T v_i(t)^2) = O(c(T))$,
            \item $\frac{1}{n} \sum_{i=1}^n \lambda_{il}^2 = O_P(c(T))$
            \item $c(T) = 0_P (\sum_{i=1}^n \lambda_{il}^2)$
            \item $\sum_{i=1}^n \lambda_{il}^2 = O_P(c(T)^2)$
            \item $c(T) = O_P( |\sum_{i=1}^n \lambda_{il}^2 - \sum_{i=k}^n \lambda_{il}^2|) $
            
        \end{enumerate}
    \end{assumption}

    \begin{assumption}
        There exists a nonincreasing function $b(T)$ sucht that as $n, T \to \infty$, the second-order differences of $v_i(t)$ satisfy:

        $$ E \left( \frac{1}{T} \sum_{t=2}^{T-1} v_i(t-1) - 2v_i(t) + v_i(t+1)^2 \right) = O(b(T))$$
    \end{assumption}

    \begin{assumption}
        There exists a nondecreasing function $d(T) \leq c(T)$ with $d(T) = o(T)$ such that $n, T \to \infty E(\frac{1}{T} \sum x_{itj}^2 = O(d(T))$ holds for all $j = 1, \ldots, \rho$. Furthermore, there is a constant $C_0 < \infty$ such that for all $\kappa \geq 1$:

        $$ E \left( \kappa_{\max} \left( \left[ \sum_{i=1}^n ( X_i ^\prime (I - \mathcal{Z}_k) X_i \right]^{-1} \right) \right) \leq C_0 \frac{1}{nT} $$
    \end{assumption}

    \begin{assumption}
        The error terms $\varepsilon_{it}$ are i.i.d with $E(\varepsilon_{it}) = 0, \operatorname{Var}(\varepsilon_{it}) = \sigma ^2 > 0$, and $E(\varepsilon_{it}^4) < \infty$. Moreover $\varepsilon_{it}$ is independent form $v_i(s)$ and $x_{isj}$ for all $t,s,j$.
    \end{assumption}



\newtheorem{theorem}{Theorem}

\begin{theorem}
Under the above assumptions, it holds that as $n, T \to \infty$:

\begin{enumerate}
    \item 
\end{enumerate}
\end{theorem}
\section{Finite Sample Properties via Simulations}

The finite sample properties of the estimators presended above are studied via Monte Carlo simulations. In addition to the KSS and Eup methods, we also consider the classical time-invariant fixed effects estimator. The setup of the simulation study is taken form \citet{kneip2012new}.

The panel-data model is given by.

\begin{equation}\label{sim_model}
y_{i t}= \sum_{j =1}^P x_{itj} \beta_j + v_i(t)+\varepsilon_{i t} \quad i=1, \ldots, n ; \quad t=1, \ldots, T
\end{equation}

we and simulate samlpes of size $n=30,100,300$ with $T = 12,30$ in a model with $P = 2$ regressors. The slope parameters' true values given by $\beta_1=\beta_2=0.5$. The regressors $X_{it} = (x_{it1}, x_{it2})^\prime$ are generated acording to a bivariate vetor autoregression model:
\begin{equation}\label{sim_x}
X_{i t}=R X_{i, t-1}+\eta_{i t} \quad \text { with } \quad R=\left(\begin{array}{cc}
0.4 & 0.05 \\
0.05 & 0.4
\end{array}\right) \quad \text { and } \quad \eta_{i t} \sim N\left(0, I_2\right)
\end{equation}

To intitalize the simulation, we set $X_{i1} \sim N(0, (I_2 - R^2)^{-1})$ and generate the rest of the sample according to (\ref{sim_x}). Thereafter, the $n$ regressor-series $\left(x_{1 i 1}, x_{2 i 1}\right)^{\prime}, \ldots,\left(x_{1 i T}, x_{2 i T}\right)^{\prime}$ are additionally shifted such that there are three different mean-value-clusters, fixed at $ \mu_1=(5,5)^{\prime}, \mu_2=(7.5,7.5)^{\prime} \text {, and } \mu_3=(10,10)^{\prime}$. This is to get a reasonable cloud of points for the regressors \citep{kneip2012new}.

We generate time-varying individual effects following five different data generating processes:
\begin{enumerate}
\item[$\text{DGP}1$:] $v_i(t)=\theta_{i0}+\theta_{i1}\frac{t}{T}+\theta_{i2}\left(\frac{t}{T}\right)^2$,
\item[$\text{DGP}2$:] $v_i(t)=\phi_i r_t$,
\item[$\text{DGP}3$:] $v_i(t)=v_{i1} \sin (\pi t / 4) +v_{i2} \cos (\pi t / 4)$,
\item[$\text{DGP}4$:] $v_i(t)=\xi_i$,
\item[$\text{DGP}5$:] $v_i(t) = v_{i1} e^{-\frac{t}{4}}  \sin (\pi t / 4)$
\end{enumerate}

where $\theta_{i j}(j=0,1,2) \sim \: \text{i.i.d.} \: 5N(0,1),\xi_i, v_{i j}(j=1,2) \sim \: \text{i.i.d.} \: 3N(0,1)$, and $r_{t+1} = r_t + \delta_t$, with $\delta_t, r_1 \sim \: \text{i.i.d.} \: 3N(0,1)$. 

The second regressor $x_{it2}$ is allowed to be endogenous with a correlation with $v_i(t)$ of $\rho = 0.5$. Let $w_{it}$ be the endogenous part of the regressor, so that $w_{it} = \rho v_i(t) + \sigma_v \sqrt{1-\rho^2}\epsilon_{it}$\footnote{In generating $w_{i t}$, the effects $v_i(t)$ is multiplied by 10 to balance with the magnitude of $x_{i t2}$.}, where $\sigma_v$ is the standard deviation of $v_i(t)$ and $\epsilon_{i t} \sim N(0,1)$. We then define the endogenous regressor as $\tilde{x}_{it2} = x_{it2} + w_{it}$.



\subsection{Baseline scenario}


\begin{table}[]
\caption{DGP 1, homoskedastic errors}
    \centering
 \begin{tabular}{lcccccc} 
\hline \multicolumn{6}{c}{MSE of Effects} \\ \hline 
$n$ & $T$ & KSS & Eup & $d_{KSS}$ & $d_{Eup}$ \\
\hline
30 & 12 &  0.7285  &  1.0413  &  2.0000  &  2.9390  \\
& 30 &  0.2749  &  0.3980  &  2.0310  &  2.4350  \\
100 & 12 &  0.3807  &  0.4259  &  2.0000  &  2.4810  \\
& 30 &  0.1754  &  0.1931  &  2.0030  &  2.2190  \\
300 & 12 &  0.2873  &  0.2912  &  2.0000  &  2.2240  \\
& 30 &  0.1484  &  0.1454  &  2.0060  &  2.2420  \\
\end{tabular} 
\begin{tabular}{ccccccc} 
\hline 
\multicolumn{7}{c}{MSE, Bias, Variance for Coefficients} \\ \hline 
& \multicolumn{3}{c}{$T=12$} & \multicolumn{3}{c}{$T=30$} \\ \cline{2-4} \cline{5-7} 
& KSS & Eup & Within & KSS & Eup & Within \\\multicolumn{7}{l}{$n = 30 } \\MSE  & 0.00395 & 0.00509 & 0.02882 & 0.00113 & 0.00141 & 0.01038\\ BIAS1  & -0.00383 & -0.00605 & -0.0097 & 0.0010 & 0.00067 & -0.00041\\ BIAS2  & -0.00107 & -0.00202 & 0.00430 & -0.00088 & -0.00099 & 0.00297\\ VAR1  & 0.06637 & 0.06303 & 0.13016 & 0.03642 & 0.03414 & 0.07439\\ VAR2  & 0.06636 & 0.06304 & 0.13008 & 0.03638 & 0.03410 & 0.07427\\ \hline 
\multicolumn{7}{l}{$n = 100 } \\MSE  & 0.00120 & 0.00127 & 0.01060 & 0.00036 & 4e-04 & 0.00400\\ BIAS1  & -0.00202 & -0.00119 & -0.00216 & 0.00063 & 0.00098 & 0.00174\\ BIAS2  & 0.00011 & 0.00064 & 0.00054 & 0.00063 & 0.00088 & 0.00121\\ VAR1  & 0.03521 & 0.03394 & 0.07055 & 0.01942 & 0.01860 & 0.04063\\ VAR2  & 0.03526 & 0.03398 & 0.07056 & 0.01941 & 0.01860 & 0.04062\\ \hline 
\multicolumn{7}{l}{$n = 300 } \\MSE  & 0.00039 & 0.00043 & 0.00337 & 0.00012 & 0.00012 & 0.00105\\ BIAS1  & -0.00109 & -0.00083 & 0.00169 & -7e-05 & -5e-05 & -0.00258\\ BIAS2  & 0.00016 & 0.00022 & -0.00129 & -0.00025 & -7e-05 & 0.00119\\ VAR1  & 0.02030 & 0.01949 & 0.04143 & 0.01118 & 0.01078 & 0.02374\\ VAR2  & 0.02031 & 0.01949 & 0.04142 & 0.01117 & 0.01079 & 0.02374\\ \hline 
\end{tabular} 

\end{table}
    
\begin{table}[]
\caption{DGP 2, homoskedastic errors}
    \centering
 \begin{tabular}{lccccccc} 
\hline \multicolumn{7}{c}{MSE of Effects} \\ \hline 
$n$ & $T$ & KSS & Bai & Eup & $d_{KSS}$ & $d_{Bai}$ & $d_{Eup}$ \\
\hline
30 & 12 &  34.6197  &  5.1895  &  0.5866  &  1.1690  &  8.0000  &  2.5970  \\
& 30 &  18.7861  &  1.6238  &  0.0748  &  1.0890  &  8.0000  &  1.0230  \\
100 & 12 &  21.1234  &  1.4683  &  0.0968  &  1.0380  &  8.0000  &  1.0060  \\
& 30 &  12.8899  &  0.4892  &  0.0433  &  1.0080  &  8.0000  &  1.0000  \\
300 & 12 &  17.9368  &  0.9283  &  0.0868  &  1.0020  &  8.0000  &  1.0000  \\
& 30 &  11.7004  &  0.3731  &  0.0370  &  1.0020  &  8.0000  &  1.0000  \\
\end{tabular} 
\begin{tabular}{ccccccccc} 
\hline 
\multicolumn{9}{c}{MSE, Bias, Variance for Coefficients} \\ \hline 
& \multicolumn{4}{c}{$T=12$} & \multicolumn{4}{c}{$T=30$} \\ \cline{2-5} \cline{6-9} 
& KSS &  Bai & Eup & Within & KSS & Bai &  Eup & Within \\\multicolumn{8}{l}{$n = 30 } \\MSE  & 0.14589 & 0.03000 & 0.00309 & 0.86332 & 0.07331 & 0.00579 & 0.00061 & 0.80906\\ BIAS1  & 0.01399 & -0.00079 & -0.0011 & 0.00225 & -0.00538 & -0.00533 & -0.00026 & 0.00223\\ BIAS2  & -0.00896 & 0.00269 & 0.00104 & 0.02220 & -0.00516 & -0.00295 & 0.00085 & 0.03484\\ VAR1  & 0.29724 & 0.12825 & 0.04154 & 0.64902 & 0.16870 & 0.03026 & 0.02340 & 0.59364\\ VAR2  & 0.29713 & 0.12781 & 0.04158 & 0.64840 & 0.16869 & 0.03022 & 0.02339 & 0.59210\\ \hline 
\multicolumn{8}{l}{$n = 100 } \\MSE  & 0.04493 & 0.00463 & 0.00045 & 0.25214 & 0.02080 & 4e-04 & 0.00015 & 0.30042\\ BIAS1  & -0.00784 & 0.00116 & 0.00023 & -0.00488 & 0.00161 & 0.00089 & -2e-04 & -0.00174\\ BIAS2  & -0.00420 & -0.00362 & 0.00027 & 0.00933 & 0.00022 & -0.00035 & 0.00014 & -0.00228\\ VAR1  & 0.15368 & 0.03662 & 0.02059 & 0.34846 & 0.08608 & 0.01391 & 0.01263 & 0.32538\\ VAR2  & 0.15356 & 0.03666 & 0.02059 & 0.34845 & 0.08606 & 0.01391 & 0.01262 & 0.32514\\ \hline 
\multicolumn{8}{l}{$n = 300 } \\MSE  & 0.01424 & 0.00135 & 0.00015 & 0.08404 & 0.00657 & 9e-05 & 5e-05 & 0.08561\\ BIAS1  & 0.00135 & 0.00158 & 0.00024 & -0.00436 & -0.00388 & -0.00034 & -0.00026 & 0.00092\\ BIAS2  & 0.00578 & -0.00037 & -0.00033 & -0.01926 & 0.00169 & 0.00037 & 5e-04 & -0.01251\\ VAR1  & 0.08518 & 0.01979 & 0.01182 & 0.20363 & 0.04741 & 0.00797 & 0.00726 & 0.19078\\ VAR2  & 0.08526 & 0.01982 & 0.01183 & 0.20366 & 0.04743 & 0.00797 & 0.00726 & 0.19081\\ \hline 
\end{tabular} 

\end{table}


\begin{table}[]
\caption{DGP 3, homoskedastic errors}
    \centering
 \begin{tabular}{lcccccc} 
\hline \multicolumn{6}{c}{MSE of Effects} \\ \hline 
$n$ & $T$ & KSS & Eup & $d_{KSS}$ & $d_{Eup}$ \\
\hline
30 & 12 &  1.4970  &  0.3913  &  2.6210  &  2.8830  \\
& 30 &  11.2204  &  0.1355  &  0.6010  &  2.0380  \\
100 & 12 &  0.5072  &  0.1895  &  2.4440  &  2.0310  \\
& 30 &  9.4474  &  0.0855  &  0.3440  &  2.0000  \\
300 & 12 &  0.3032  &  0.1728  &  2.2920  &  2.0000  \\
& 30 &  9.0916  &  0.0732  &  0.1000  &  2.0000  \\
\end{tabular} 
\begin{tabular}{ccccccc} 
\hline 
\multicolumn{7}{c}{MSE, Bias, Variance for Coefficients} \\ \hline 
& \multicolumn{3}{c}{$T=12$} & \multicolumn{3}{c}{$T=30$} \\ \cline{2-4} \cline{5-7} 
& KSS & Eup & Within & KSS & Eup & Within \\\multicolumn{7}{l}{$n = 30 } \\MSE  & 0.00976 & 0.00227 & 0.04683 & 0.01711 & 0.00059 & 0.01575\\ BIAS1  & -0.00549 & -0.00149 & 0.00992 & -0.00299 & -6e-05 & -0.00285\\ BIAS2  & -0.00303 & 0.00064 & -0.00404 & 0.00182 & -0.00011 & 0.00175\\ VAR1  & 0.09323 & 0.04106 & 0.17103 & 0.09263 & 0.02373 & 0.10165\\ VAR2  & 0.09308 & 0.04105 & 0.17108 & 0.09260 & 0.02374 & 0.10166\\ \hline 
\multicolumn{7}{l}{$n = 100 } \\MSE  & 0.00265 & 0.00048 & 0.01203 & 0.00424 & 0.00017 & 0.00422\\ BIAS1  & 0.0014 & 0.00018 & 0.00261 & 0.00016 & 0.00023 & -0.00051\\ BIAS2  & 0.00134 & -4e-04 & 0.00299 & 0.00037 & -0.00019 & -0.00016\\ VAR1  & 0.04930 & 0.02112 & 0.09308 & 0.04502 & 0.01265 & 0.05495\\ VAR2  & 0.04927 & 0.02113 & 0.09295 & 0.04505 & 0.01265 & 0.05497\\ \hline 
\multicolumn{7}{l}{$n = 300 } \\MSE  & 0.00100 & 0.00016 & 0.00451 & 0.00096 & 5e-05 & 0.00134\\ BIAS1  & 0.00231 & -0.00095 & 0.00208 & 0.00087 & -0.00041 & 0.00084\\ BIAS2  & 0.00033 & 0.00085 & 0.00158 & -0.00232 & 0.00036 & -0.00282\\ VAR1  & 0.02838 & 0.01209 & 0.05424 & 0.02328 & 0.00725 & 0.03216\\ VAR2  & 0.02842 & 0.01208 & 0.05428 & 0.02329 & 0.00725 & 0.03216\\ \hline 
\end{tabular} 

\end{table}


\begin{table}[]
\caption{DGP 4, homoskedastic errors}
    \centering
 \begin{tabular}{lccccccccc} 
\hline \multicolumn{8}{c}{MSE of the Time-Varying Individual Effects} \\ \hline 
$n$ & $T$ & KSS & $ \text{Bai}_{\hat{d} = 8}$ & $\text{Bai}_{\hat{d} = d}$ & Eup & $\hat{d}_{KSS}$ & $\hat{d}_{Eup}$ \\
\hline
30 & 12 &  0.5544  &  1.5392  &  0.6144  &  1.0360  &  1.0000  &  2.5890  \\
& 30 &  0.1573  &  0.9380  &  0.2299  &  0.2351  &  1.0000  &  1.0090  \\
100 & 12 &  0.2017  &  0.9803  &  0.1938  &  0.1979  &  1.0000  &  1.0070  \\
& 30 &  0.0723  &  0.5173  &  0.0799  &  0.0805  &  1.0000  &  1.0000  \\
300 & 12 &  0.1277  &  0.7990  &  0.1202  &  0.1213  &  1.0000  &  1.0000  \\
& 30 &  0.0464  &  0.3890  &  0.0485  &  0.0488  &  1.0000  &  1.0000  \\
\end{tabular} 
\begin{tabular}{ccccccccccc} 
\hline 
\multicolumn{10}{c}{MSE, Bias, Variance and Power for the Common Slope Coefficients} \\ \hline 
& \multicolumn{5}{c}{$T=12$} & \multicolumn{5}{c}{$T=30$} \\ \cline{2-6} \cline{7-11} 
& KSS & $ \text{Bai}_{\hat{d} = 8}$ & $\text{Bai}_{\hat{d} = d}$& Eup & Within & KSS & \text{Bai}_{\hat{d} = 8} & \text{Bai}_{\hat{d} = d} & Eup & Within \\\multicolumn{8}{l}{$n = 30 } \\$\text{MSE}_\hat{\beta}$ & 0.0037 & 0.0126 & 0.0035 & 0.0049 & 0.0030 & 0.0010 & 0.0025 & 0.0012 & 0.0012 & 0.0010\\Bias $\hat{\beta}_1$ & -0.0021 & -0.0032 & 0.0003 & -0.0007 & -0.0018 & 0.0000 & 0.0002 & 0.0000 & 0.0000 & 0.0002\\Bias $\hat{\beta}_2$ & 0.0021 & 0.0029 & 0.0032 & 0.0012 & 0.0021 & 0.0007 & 0.0005 & 0.0007 & 0.0008 & 0.0008\\$\text{Var}(\hat{\beta}_1)$ & 0.0604 & 0.1343 & 0.0503 & 0.0524 & 0.0535 & 0.0337 & 0.0348 & 0.0301 & 0.0301 & 0.0318\\$\text{Var}(\hat{\beta}_2)$ & 0.0603 & 0.1343 & 0.0503 & 0.0524 & 0.0534 & 0.0337 & 0.0348 & 0.0301 & 0.0301 & 0.0318\\Power $\hat{\beta}_1$ & 1.0000 & 0.9710 & 1.0000 & 1.0000 & 1.0000 & 1.0000 & 1.0000 & 1.0000 & 1.0000 & 1.0000\\Power $\hat{\beta}_2$ & 1.0000 & 0.9730 & 1.0000 & 1.0000 & 1.0000 & 1.0000 & 1.0000 & 1.0000 & 1.0000 & 1.0000\\ \hline 
\multicolumn{8}{l}{$n = 100 } \\$\text{MSE}_\hat{\beta}$ & 0.0011 & 0.0027 & 0.0009 & 0.0009 & 0.0009 & 0.0003 & 0.0005 & 0.0003 & 0.0003 & 0.0003\\Bias $\hat{\beta}_1$ & 0.0005 & -0.0003 & -0.0006 & -0.0006 & -0.0001 & 0.0003 & 0.0004 & 0.0002 & 0.0002 & 0.0003\\Bias $\hat{\beta}_2$ & 0.0014 & -0.0014 & 0.0003 & 0.0002 & 0.0008 & -0.0001 & 0.0004 & -0.0001 & -0.0001 & -0.0001\\$\text{Var}(\hat{\beta}_1)$ & 0.0324 & 0.0446 & 0.0283 & 0.0283 & 0.0289 & 0.0181 & 0.0183 & 0.0169 & 0.0169 & 0.0172\\$\text{Var}(\hat{\beta}_2)$ & 0.0324 & 0.0447 & 0.0283 & 0.0283 & 0.0289 & 0.0181 & 0.0183 & 0.0169 & 0.0169 & 0.0172\\Power $\hat{\beta}_1$ & 1.0000 & 1.0000 & 1.0000 & 1.0000 & 1.0000 & 1.0000 & 1.0000 & 1.0000 & 1.0000 & 1.0000\\Power $\hat{\beta}_2$ & 1.0000 & 1.0000 & 1.0000 & 1.0000 & 1.0000 & 1.0000 & 1.0000 & 1.0000 & 1.0000 & 1.0000\\ \hline 
\multicolumn{8}{l}{$n = 300 } \\$\text{MSE}_\hat{\beta}$ & 0.0004 & 0.0008 & 0.0003 & 0.0003 & 0.0003 & 0.0001 & 0.0002 & 0.0001 & 0.0001 & 0.0001\\Bias $\hat{\beta}_1$ & -0.0008 & -0.0031 & -0.0007 & -0.0007 & -0.0007 & 0.0004 & 0.0006 & 0.0005 & 0.0005 & 0.0005\\Bias $\hat{\beta}_2$ & 0.0025 & 0.0042 & 0.0023 & 0.0023 & 0.0021 & 0.0002 & 0.0012 & 0.0004 & 0.0004 & 0.0004\\$\text{Var}(\hat{\beta}_1)$ & 0.0187 & 0.0261 & 0.0167 & 0.0167 & 0.0168 & 0.0105 & 0.0109 & 0.0100 & 0.0100 & 0.0100\\$\text{Var}(\hat{\beta}_2)$ & 0.0187 & 0.0261 & 0.0167 & 0.0167 & 0.0169 & 0.0104 & 0.0109 & 0.0099 & 0.0099 & 0.0100\\Power $\hat{\beta}_1$ & 1.0000 & 1.0000 & 1.0000 & 1.0000 & 1.0000 & 1.0000 & 1.0000 & 1.0000 & 1.0000 & 1.0000\\Power $\hat{\beta}_2$ & 1.0000 & 1.0000 & 1.0000 & 1.0000 & 1.0000 & 1.0000 & 1.0000 & 1.0000 & 1.0000 & 1.0000\\ \hline 
\end{tabular} 

\end{table}


\begin{table}[]
\caption{DGP 5, homoskedastic errors}
    \centering
 \begin{tabular}{lcccccc} 
\hline \multicolumn{6}{c}{MSE of Effects} \\ \hline 
$n$ & $T$ & KSS & Eup & $d_{KSS}$ & $d_{Eup}$ \\
\hline
30 & 12 &  0.8526  &  0.3899  &  1.4940  &  2.5420  \\
& 30 &  0.3354  &  0.0799  &  1.4380  &  1.0140  \\
100 & 12 &  0.3683  &  0.0973  &  1.3370  &  1.0080  \\
& 30 &  0.1723  &  0.0467  &  1.2950  &  1.0000  \\
300 & 12 &  0.2238  &  0.0874  &  1.1200  &  1.0000  \\
& 30 &  0.1206  &  0.0376  &  1.0860  &  1.0000  \\
\end{tabular} 
\begin{tabular}{ccccccc} 
\hline 
\multicolumn{7}{c}{MSE, Bias, Variance for Coefficients} \\ \hline 
& \multicolumn{3}{c}{$T=12$} & \multicolumn{3}{c}{$T=30$} \\ \cline{2-4} \cline{5-7} 
& KSS & Eup & Within & KSS & Eup & Within \\\multicolumn{7}{l}{$n = 30 } \\MSE  & 0.00556 & 0.00222 & 0.00625 & 0.00162 & 0.00057 & 0.00146\\ BIAS1  & -0.00198 & 0.00044 & 0.00128 & 2e-05 & 0.00061 & 0.00059\\ BIAS2  & -0.00103 & -3e-04 & 0.00059 & -0.00365 & -0.00099 & -0.00291\\ VAR1  & 0.07115 & 0.03826 & 0.06868 & 0.03867 & 0.02272 & 0.03583\\ VAR2  & 0.07130 & 0.03826 & 0.06877 & 0.03862 & 0.02273 & 0.03581\\ \hline 
\multicolumn{7}{l}{$n = 100 } \\MSE  & 0.00151 & 0.00042 & 0.00175 & 0.00044 & 0.00016 & 0.00044\\ BIAS1  & 0.00124 & -0.00016 & 0.00069 & 0.00051 & 0.00011 & -0.00033\\ BIAS2  & 0.00201 & 0.00012 & 0.00055 & 0.00062 & -0.00016 & 0.00063\\ VAR1  & 0.03774 & 0.01981 & 0.03698 & 0.02080 & 0.01232 & 0.01938\\ VAR2  & 0.03778 & 0.01980 & 0.03702 & 0.02079 & 0.01232 & 0.01938\\ \hline 
\multicolumn{7}{l}{$n = 300 } \\MSE  & 0.00053 & 0.00013 & 6e-04 & 0.00015 & 5e-05 & 0.00015\\ BIAS1  & 0.00132 & 0.00065 & 0.00107 & 0.00061 & 0.00016 & 0.00076\\ BIAS2  & -0.00081 & -0.00078 & -0.00085 & 7e-04 & -7e-05 & 3e-04\\ VAR1  & 0.02172 & 0.01138 & 0.02166 & 0.01198 & 0.00708 & 0.01130\\ VAR2  & 0.02172 & 0.01139 & 0.02167 & 0.01199 & 0.00708 & 0.01130\\ \hline 
\end{tabular} 

\end{table}

\subsection{Heteroskedastic error terms}


\begin{table}[]
\caption{DGP 1, heteroskedastic errors}
    \centering
 \begin{tabular}{lcccccc} 
\hline \multicolumn{6}{c}{MSE of Effects} \\ \hline 
$n$ & $T$ & KSS & Eup & $d_{KSS}$ & $d_{Eup}$ \\
\hline
30 & 12 &  62.2008  &  103.9136  &  1.0150  &  2.9020  \\
& 30 &  22.7314  &  54.9450  &  1.0920  &  2.1800  \\
100 & 12 &  30.0534  &  26.4260  &  1.0060  &  1.1400  \\
& 30 &  11.5718  &  12.6986  &  1.0340  &  1.0000  \\
300 & 12 &  18.7497  &  15.6618  &  1.0060  &  1.0000  \\
& 30 &  8.7833  &  8.5119  &  1.0080  &  1.0000  \\
\end{tabular} 
\begin{tabular}{ccccccc} 
\hline 
\multicolumn{7}{c}{MSE, Bias, Variance for Coefficients} \\ \hline 
& \multicolumn{3}{c}{$T=12$} & \multicolumn{3}{c}{$T=30$} \\ \cline{2-4} \cline{5-7} 
& KSS & Eup & Within & KSS & Eup & Within \\\multicolumn{7}{l}{$n = 30 } \\MSE  & 0.45662 & 0.47022 & 0.39581 & 0.13916 & 0.17615 & 0.14128\\ BIAS1  & 0.02107 & -0.10011 & 0.02175 & 0.00326 & -0.08382 & -0.00148\\ BIAS2  & 0.01217 & -0.10747 & 0.01074 & -0.01036 & -0.08419 & -0.00533\\ VAR1  & 0.68428 & 0.24212 & 0.60944 & 0.38003 & 0.16653 & 0.36053\\ VAR2  & 0.68449 & 0.24222 & 0.60872 & 0.38013 & 0.16567 & 0.36087\\ \hline 
\multicolumn{7}{l}{$n = 100 } \\MSE  & 0.12850 & 0.09204 & 0.11475 & 0.03595 & 0.03302 & 0.03596\\ BIAS1  & 0.00169 & 0.00888 & 0.00601 & -0.00524 & -0.01039 & -0.00635\\ BIAS2  & -0.01807 & -0.01504 & -0.01958 & 0.00782 & 0.00547 & 0.00904\\ VAR1  & 0.36602 & 0.12415 & 0.32788 & 0.20460 & 0.08674 & 0.19520\\ VAR2  & 0.36613 & 0.12540 & 0.32791 & 0.20484 & 0.08630 & 0.19544\\ \hline 
\multicolumn{7}{l}{$n = 300 } \\MSE  & 0.04230 & 0.02423 & 0.03796 & 0.01365 & 0.01037 & 0.01403\\ BIAS1  & -0.01408 & -0.0146 & -0.01769 & 0.00177 & 0.00143 & 0.00193\\ BIAS2  & 0.01154 & 0.01231 & 0.01038 & -0.00453 & -0.00318 & -0.00131\\ VAR1  & 0.21112 & 0.07088 & 0.19153 & 0.11820 & 0.04929 & 0.11391\\ VAR2  & 0.21122 & 0.07081 & 0.19151 & 0.11812 & 0.04938 & 0.11382\\ \hline 
\end{tabular} 

\end{table}
    
\begin{table}[]
\caption{DGP 2, heteroskedastic errors}
    \centering
 \begin{tabular}{lcccccc} 
\hline \multicolumn{6}{c}{MSE of Effects} \\ \hline 
$n$ & $T$ & KSS & Eup & $d_{KSS}$ & $d_{Eup}$ \\
\hline
30 & 12 &  123.2147  &  63.0828  &  1.2010  &  2.9100  \\
& 30 &  67.7417  &  24.2219  &  1.2320  &  1.9840  \\
100 & 12 &  62.2112  &  13.7073  &  1.0850  &  1.0790  \\
& 30 &  44.2181  &  5.5160  &  1.0790  &  1.0000  \\
300 & 12 &  48.3361  &  10.9871  &  1.0120  &  1.0000  \\
& 30 &  37.7969  &  4.6324  &  1.0120  &  1.0000  \\
\end{tabular} 
\begin{tabular}{ccccccc} 
\hline 
\multicolumn{7}{c}{MSE, Bias, Variance for Coefficients} \\ \hline 
& \multicolumn{3}{c}{$T=12$} & \multicolumn{3}{c}{$T=30$} \\ \cline{2-4} \cline{5-7} 
& KSS & Eup & Within & KSS & Eup & Within \\\multicolumn{7}{l}{$n = 30 } \\MSE  & 0.68070 & 0.28860 & 1.25152 & 0.25940 & 0.08121 & 0.86986\\ BIAS1  & 0.0328 & -0.02384 & 0.05872 & -0.01378 & 0.00062 & -0.04395\\ BIAS2  & 0.01928 & -0.00630 & 0.06895 & 0.01087 & -0.00281 & -0.00743\\ VAR1  & 0.81379 & 0.22700 & 0.91104 & 0.49003 & 0.15226 & 0.69982\\ VAR2  & 0.81469 & 0.22646 & 0.91139 & 0.49007 & 0.15186 & 0.69979\\ \hline 
\multicolumn{7}{l}{$n = 100 } \\MSE  & 0.19950 & 0.05706 & 0.40377 & 0.07818 & 0.02095 & 0.31214\\ BIAS1  & 0.0073 & 0.00018 & -0.00691 & -0.01272 & 0.00268 & -0.00511\\ BIAS2  & -0.00089 & -0.00042 & 0.01344 & -0.01231 & -0.00311 & -0.00672\\ VAR1  & 0.42915 & 0.11487 & 0.48387 & 0.26185 & 0.07900 & 0.37229\\ VAR2  & 0.43006 & 0.11496 & 0.48515 & 0.26174 & 0.07884 & 0.37176\\ \hline 
\multicolumn{7}{l}{$n = 300 } \\MSE  & 0.07165 & 0.01871 & 0.14050 & 0.02757 & 0.00657 & 0.09870\\ BIAS1  & -0.01665 & -0.01419 & -0.00914 & -0.00748 & -0.00514 & 0.01598\\ BIAS2  & 0.00195 & 0.01368 & 0.00258 & 0.00171 & 0.00555 & -0.00396\\ VAR1  & 0.24957 & 0.06747 & 0.28511 & 0.15092 & 0.04613 & 0.21874\\ VAR2  & 0.24957 & 0.06752 & 0.28531 & 0.15105 & 0.04614 & 0.21898\\ \hline 
\end{tabular} 

\end{table}


\begin{table}[]
\caption{DGP 3, heteroskedastic errors}
    \centering
 \begin{tabular}{lcccccc} 
\hline \multicolumn{6}{c}{MSE of Effects} \\ \hline 
$n$ & $T$ & KSS & Eup & $d_{KSS}$ & $d_{Eup}$ \\
\hline
30 & 12 &  23.2244  &  11.5184  &  1.2420  &  2.9520  \\
& 30 &  16.2401  &  7.6352  &  0.7390  &  2.1840  \\
100 & 12 &  21.4727  &  11.8651  &  0.6620  &  0.8620  \\
& 30 &  13.5124  &  8.6707  &  0.6180  &  0.3470  \\
300 & 12 &  23.4724  &  8.9875  &  0.5860  &  0.0000  \\
& 30 &  14.9862  &  9.0223  &  0.6260  &  0.0000  \\
\end{tabular} 
\begin{tabular}{ccccccc} 
\hline 
\multicolumn{7}{c}{MSE, Bias, Variance for Coefficients} \\ \hline 
& \multicolumn{3}{c}{$T=12$} & \multicolumn{3}{c}{$T=30$} \\ \cline{2-4} \cline{5-7} 
& KSS & Eup & Within & KSS & Eup & Within \\\multicolumn{7}{l}{$n = 30 } \\MSE  & 0.12051 & 0.05742 & 0.09927 & 0.04608 & 0.01973 & 0.04339\\ BIAS1  & 0.00671 & -0.01423 & 0.0049 & -0.00701 & -0.00485 & -0.00423\\ BIAS2  & 0.00201 & -0.00245 & -0.00208 & 0.00859 & -0.00141 & 0.01224\\ VAR1  & 0.35849 & 0.16233 & 0.30339 & 0.19652 & 0.12426 & 0.20395\\ VAR2  & 0.35877 & 0.16228 & 0.30364 & 0.19653 & 0.12428 & 0.20392\\ \hline 
\multicolumn{7}{l}{$n = 100 } \\MSE  & 0.07425 & 0.02745 & 0.06404 & 0.02649 & 0.01242 & 0.02547\\ BIAS1  & 0.00444 & -0.00334 & 0.00232 & 0.00372 & -0.00041 & 0.00388\\ BIAS2  & 0.00826 & 0.00240 & 0.00226 & 0.00102 & -3e-05 & 0.00069\\ VAR1  & 0.23135 & 0.15214 & 0.23602 & 0.13795 & 0.10428 & 0.15011\\ VAR2  & 0.23156 & 0.15218 & 0.23628 & 0.13794 & 0.10431 & 0.14997\\ \hline 
\multicolumn{7}{l}{$n = 300 } \\MSE  & 0.05113 & 0.02069 & 0.04826 & 0.01971 & 0.00899 & 0.01921\\ BIAS1  & -0.01414 & -0.01297 & -0.00947 & -0.00911 & -0.01009 & -0.00842\\ BIAS2  & 0.01296 & 0.01215 & 0.01251 & 0.01185 & 0.00963 & 0.01177\\ VAR1  & 0.20095 & 0.14298 & 0.21781 & 0.12167 & 0.09304 & 0.13317\\ VAR2  & 0.20096 & 0.14294 & 0.21790 & 0.12167 & 0.09305 & 0.13326\\ \hline 
\end{tabular} 

\end{table}


\begin{table}[]
\caption{DGP 4,heteroskedastic errors}
    \centering
 \begin{tabular}{lcccccc} 
\hline \multicolumn{6}{c}{MSE of Effects} \\ \hline 
$n$ & $T$ & KSS & Eup & $d_{KSS}$ & $d_{Eup}$ \\
\hline
30 & 12 &  61.4926  &  88.6868  &  0.9600  &  2.8970  \\
& 30 &  20.7066  &  39.5100  &  0.9980  &  1.7540  \\
100 & 12 &  26.1465  &  17.5762  &  0.9920  &  0.5200  \\
& 30 &  9.0335  &  11.7426  &  1.0000  &  0.6010  \\
300 & 12 &  15.8298  &  11.4628  &  1.0000  &  0.2080  \\
& 30 &  5.8010  &  8.5898  &  1.0000  &  0.6320  \\
\end{tabular} 
\begin{tabular}{ccccccc} 
\hline 
\multicolumn{7}{c}{MSE, Bias, Variance for Coefficients} \\ \hline 
& \multicolumn{3}{c}{$T=12$} & \multicolumn{3}{c}{$T=30$} \\ \cline{2-4} \cline{5-7} 
& KSS & Eup & Within & KSS & Eup & Within \\\multicolumn{7}{l}{$n = 30 } \\MSE  & 0.40418 & 0.36742 & 0.35195 & 0.14026 & 0.13204 & 0.12766\\ BIAS1  & 0.00923 & -0.10632 & 0.00545 & 0.01019 & -0.06906 & 0.01319\\ BIAS2  & -0.00492 & -0.10176 & 0.00190 & -0.00178 & -0.06864 & 0.00037\\ VAR1  & 0.66152 & 0.23434 & 0.59728 & 0.37373 & 0.16055 & 0.35302\\ VAR2  & 0.66217 & 0.23521 & 0.59806 & 0.37424 & 0.16165 & 0.35383\\ \hline 
\multicolumn{7}{l}{$n = 100 } \\MSE  & 0.13538 & 0.07908 & 0.11030 & 0.04071 & 0.03514 & 0.03699\\ BIAS1  & 0.00523 & -0.00705 & -0.00182 & 0.00372 & -0.02577 & 0.00376\\ BIAS2  & -0.00323 & -0.01962 & -0.00552 & 0.00864 & -0.02304 & 0.00852\\ VAR1  & 0.35937 & 0.12077 & 0.32193 & 0.20200 & 0.08494 & 0.19172\\ VAR2  & 0.35907 & 0.12062 & 0.32163 & 0.20199 & 0.08474 & 0.19164\\ \hline 
\multicolumn{7}{l}{$n = 300 } \\MSE  & 0.05004 & 0.02695 & 0.04076 & 0.01244 & 0.01391 & 0.01081\\ BIAS1  & 0.0181 & 0.0060 & 0.01812 & -0.00027 & -0.00499 & 0.00086\\ BIAS2  & -0.00858 & -0.01660 & -0.00258 & -0.00276 & -0.00453 & -0.00018\\ VAR1  & 0.20842 & 0.07155 & 0.18807 & 0.11660 & 0.04944 & 0.11184\\ VAR2  & 0.20843 & 0.07146 & 0.18801 & 0.11661 & 0.04956 & 0.11186\\ \hline 
\end{tabular} 

\end{table}


\begin{table}[]
\caption{DGP 5, heteroskedastic errors}
    \centering
 \begin{tabular}{lcccccc} 
\hline \multicolumn{6}{c}{MSE of Effects} \\ \hline 
$n$ & $T$ & KSS & Eup & $d_{KSS}$ & $d_{Eup}$ \\
\hline
30 & 12 &  62.9455  &  72.7394  &  0.8030  &  2.8040  \\
& 30 &  19.4050  &  14.1902  &  0.8150  &  0.8160  \\
100 & 12 &  24.4413  &  1.5621  &  0.7640  &  0.0370  \\
& 30 &  8.1026  &  0.2732  &  0.7660  &  0.0000  \\
300 & 12 &  12.2654  &  0.6749  &  0.6640  &  0.0000  \\
& 30 &  4.7554  &  0.2700  &  0.6860  &  0.0000  \\
\end{tabular} 
\begin{tabular}{ccccccc} 
\hline 
\multicolumn{7}{c}{MSE, Bias, Variance for Coefficients} \\ \hline 
& \multicolumn{3}{c}{$T=12$} & \multicolumn{3}{c}{$T=30$} \\ \cline{2-4} \cline{5-7} 
& KSS & Eup & Within & KSS & Eup & Within \\\multicolumn{7}{l}{$n = 30 } \\MSE  & 0.45710 & 0.30831 & 0.37179 & 0.14152 & 0.07563 & 0.12816\\ BIAS1  & 0.01187 & -0.02646 & 0.00954 & -0.01295 & -0.01393 & -0.00843\\ BIAS2  & -0.01628 & -0.07574 & -0.01176 & 0.00408 & -0.00674 & 0.00420\\ VAR1  & 0.61686 & 0.21771 & 0.59636 & 0.35137 & 0.14614 & 0.35458\\ VAR2  & 0.61663 & 0.21821 & 0.59872 & 0.35155 & 0.14573 & 0.35445\\ \hline 
\multicolumn{7}{l}{$n = 100 } \\MSE  & 0.12906 & 0.04897 & 0.10661 & 0.04056 & 0.02069 & 0.03867\\ BIAS1  & 0.00693 & -0.00611 & 0.00268 & 0.00284 & 0.00564 & 0.00317\\ BIAS2  & 0.00524 & -0.00018 & 0.00690 & -0.00618 & -0.00433 & -0.00714\\ VAR1  & 0.32567 & 0.11390 & 0.32164 & 0.18640 & 0.07928 & 0.19187\\ VAR2  & 0.32592 & 0.11374 & 0.32172 & 0.18655 & 0.07921 & 0.19197\\ \hline 
\multicolumn{7}{l}{$n = 300 } \\MSE  & 0.04004 & 0.01766 & 0.03624 & 0.01274 & 0.00597 & 0.01228\\ BIAS1  & -0.00479 & -0.00556 & -0.00078 & 0.00434 & 0.00104 & 0.00283\\ BIAS2  & 0.00684 & 0.00804 & 0.00254 & 0.00133 & -1e-04 & 0.00075\\ VAR1  & 0.17942 & 0.06641 & 0.18825 & 0.10442 & 0.04563 & 0.11188\\ VAR2  & 0.17941 & 0.06640 & 0.18808 & 0.10444 & 0.04562 & 0.11190\\ \hline 
\end{tabular} 

\end{table}

\subsection{Weakly autocrorrelated error terms}


\begin{table}[]
\caption{DGP 1, weakly autocorrelated errors}
    \centering
 \begin{tabular}{lcccccc} 
\hline \multicolumn{6}{c}{MSE of Effects} \\ \hline 
$n$ & $T$ & KSS & Eup & $d_{KSS}$ & $d_{Eup}$ \\
\hline
30 & 12 &  0.7978  &  1.0860  &  2.0820  &  2.9960  \\
& 30 &  0.3459  &  0.5496  &  2.8500  &  3.2440  \\
100 & 12 &  0.3820  &  0.4205  &  2.0000  &  2.4780  \\
& 30 &  0.2036  &  0.2386  &  2.8690  &  2.8020  \\
300 & 12 &  0.2789  &  0.2785  &  2.0000  &  2.1280  \\
& 30 &  0.1497  &  0.1472  &  2.0000  &  2.2480  \\
\end{tabular} 
\begin{tabular}{ccccccc} 
\hline 
\multicolumn{7}{c}{MSE, Bias, Variance for Coefficients} \\ \hline 
& \multicolumn{3}{c}{$T=12$} & \multicolumn{3}{c}{$T=30$} \\ \cline{2-4} \cline{5-7} 
& KSS & Eup & Within & KSS & Eup & Within \\\multicolumn{7}{l}{$n = 30 } \\MSE  & 0.00370 & 0.00485 & 0.02985 & 0.00133 & 0.00170 & 0.01128\\ BIAS1  & 0.00048 & 0.00113 & 0.00028 & -0.00174 & -0.00067 & 0.00368\\ BIAS2  & -0.00185 & -0.00317 & -0.00398 & -0.00177 & -0.00156 & 0.00319\\ VAR1  & 0.06589 & 0.04051 & 0.12898 & 0.03853 & 0.02744 & 0.07448\\ VAR2  & 0.06611 & 0.04047 & 0.12953 & 0.03856 & 0.02744 & 0.07457\\ \hline 
\multicolumn{7}{l}{$n = 100 } \\MSE  & 0.00118 & 0.00129 & 0.01003 & 0.00037 & 0.00039 & 0.00384\\ BIAS1  & -0.00125 & -0.00069 & 0.00225 & 0.0000 & -0.00076 & -0.00382\\ BIAS2  & -0.00084 & -0.00045 & -0.00115 & 0.00026 & -7e-05 & -1e-04\\ VAR1  & 0.03516 & 0.02448 & 0.07083 & 0.01965 & 0.01583 & 0.04045\\ VAR2  & 0.03521 & 0.02447 & 0.07080 & 0.01965 & 0.01582 & 0.04044\\ \hline 
\multicolumn{7}{l}{$n = 300 } \\MSE  & 0.00038 & 0.00037 & 0.00317 & 0.00011 & 0.00012 & 0.00120\\ BIAS1  & -0.00083 & -0.00084 & -0.00075 & 0.00039 & 0.00031 & 0.00079\\ BIAS2  & -0.00131 & -0.00148 & -0.00230 & -0.00085 & -0.00078 & -0.00238\\ VAR1  & 0.02034 & 0.01442 & 0.04136 & 0.01118 & 0.00925 & 0.02370\\ VAR2  & 0.02034 & 0.01442 & 0.04137 & 0.01117 & 0.00924 & 0.02369\\ \hline 
\end{tabular} 

\end{table}
    
\begin{table}[]
\caption{DGP 2, weakly autocorrelated errors}
    \centering
 \begin{tabular}{lccccccccc} 
\hline \multicolumn{8}{c}{MSE of the Time-Varying Individual Effects} \\ \hline 
$n$ & $T$ & KSS & $ \text{Bai}_{\hat{d} = 8}$ & $\text{Bai}_{\hat{d} = d}$ & Eup & $\hat{d}_{KSS}$ & $\hat{d}_{Eup}$ \\
\hline
30 & 12 &  36.6275  &  4.6366  &  0.1224  &  0.1580  &  1.1930  &  1.0170  \\
& 30 &  18.4334  &  1.7280  &  0.0697  &  0.0711  &  1.0930  &  1.0010  \\
100 & 12 &  21.9760  &  1.3175  &  0.0961  &  0.0961  &  1.0540  &  1.0000  \\
& 30 &  12.8561  &  0.5144  &  0.0450  &  0.0450  &  1.0120  &  1.0000  \\
300 & 12 &  20.3633  &  0.8869  &  0.0898  &  0.0898  &  0.9980  &  1.0000  \\
& 30 &  10.9710  &  0.4058  &  0.0383  &  0.0383  &  1.0020  &  1.0000  \\
\end{tabular} 
\begin{tabular}{ccccccccccc} 
\hline 
\multicolumn{10}{c}{MSE, Bias, Variance and Power for the Common Slope Coefficients} \\ \hline 
& \multicolumn{5}{c}{$T=12$} & \multicolumn{5}{c}{$T=30$} \\ \cline{2-6} \cline{7-11} 
& KSS & $ \text{Bai}_{\hat{d} = 8}$ & $\text{Bai}_{\hat{d} = d}$& Eup & Within & KSS & \text{Bai}_{\hat{d} = 8} & \text{Bai}_{\hat{d} = d} & Eup & Within \\\multicolumn{8}{l}{$n = 30 } \\$\text{MSE}_\hat{\beta}$ & 0.1718 & 0.0268 & 0.0016 & 0.0018 & 0.8705 & 0.0709 & 0.0057 & 0.0006 & 0.0006 & 0.9498\\Bias $\hat{\beta}_1$ & -0.0077 & 0.0034 & -0.0001 & -0.0009 & -0.0326 & -0.0235 & -0.0020 & -0.0002 & -0.0003 & -0.0088\\Bias $\hat{\beta}_2$ & 0.0115 & -0.0021 & 0.0002 & -0.0002 & -0.0146 & 0.0000 & -0.0034 & -0.0004 & -0.0004 & -0.0857\\$\text{Var}(\hat{\beta}_1)$ & 0.2998 & 0.0300 & 0.0297 & 0.0299 & 0.8062 & 0.1697 & 0.0190 & 0.0202 & 0.0202 & 0.8301\\$\text{Var}(\hat{\beta}_2)$ & 0.3004 & 0.0302 & 0.0297 & 0.0299 & 0.8169 & 0.1695 & 0.0190 & 0.0202 & 0.0202 & 0.8188\\Power $\hat{\beta}_1$ & 0.4700 & 0.9970 & 1.0000 & 1.0000 & 0.1520 & 0.7210 & 0.9970 & 1.0000 & 1.0000 & 0.1700\\Power $\hat{\beta}_2$ & 0.4910 & 0.9960 & 1.0000 & 1.0000 & 0.1580 & 0.7490 & 0.9960 & 1.0000 & 1.0000 & 0.1380\\ \hline 
\multicolumn{8}{l}{$n = 100 } \\$\text{MSE}_\hat{\beta}$ & 0.0470 & 0.0035 & 0.0005 & 0.0005 & 0.2812 & 0.0201 & 0.0004 & 0.0002 & 0.0002 & 0.2936\\Bias $\hat{\beta}_1$ & -0.0043 & 0.0007 & 0.0004 & 0.0004 & -0.0199 & -0.0020 & 0.0000 & 0.0006 & 0.0006 & 0.0060\\Bias $\hat{\beta}_2$ & 0.0022 & -0.0024 & -0.0007 & -0.0007 & 0.0089 & -0.0003 & -0.0009 & -0.0005 & -0.0005 & -0.0182\\$\text{Var}(\hat{\beta}_1)$ & 0.1556 & 0.0176 & 0.0163 & 0.0163 & 0.4551 & 0.0863 & 0.0102 & 0.0110 & 0.0110 & 0.4746\\$\text{Var}(\hat{\beta}_2)$ & 0.1558 & 0.0175 & 0.0163 & 0.0163 & 0.4569 & 0.0863 & 0.0102 & 0.0110 & 0.0110 & 0.4788\\Power $\hat{\beta}_1$ & 0.8010 & 1.0000 & 1.0000 & 1.0000 & 0.2930 & 0.9800 & 1.0000 & 1.0000 & 1.0000 & 0.2870\\Power $\hat{\beta}_2$ & 0.8020 & 1.0000 & 1.0000 & 1.0000 & 0.3000 & 0.9800 & 1.0000 & 1.0000 & 1.0000 & 0.2710\\ \hline 
\multicolumn{8}{l}{$n = 300 } \\$\text{MSE}_\hat{\beta}$ & 0.0163 & 0.0009 & 0.0002 & 0.0002 & 0.0742 & 0.0065 & 0.0001 & 0.0001 & 0.0001 & 0.0957\\Bias $\hat{\beta}_1$ & -0.0010 & 0.0020 & -0.0009 & -0.0009 & 0.0196 & -0.0024 & -0.0006 & -0.0001 & -0.0001 & 0.0035\\Bias $\hat{\beta}_2$ & 0.0024 & -0.0005 & 0.0012 & 0.0012 & 0.0154 & 0.0043 & 0.0002 & 0.0001 & 0.0001 & 0.0118\\$\text{Var}(\hat{\beta}_1)$ & 0.0889 & 0.0109 & 0.0094 & 0.0094 & 0.2663 & 0.0469 & 0.0062 & 0.0064 & 0.0064 & 0.2784\\$\text{Var}(\hat{\beta}_2)$ & 0.0889 & 0.0109 & 0.0095 & 0.0095 & 0.2638 & 0.0469 & 0.0062 & 0.0064 & 0.0064 & 0.2761\\Power $\hat{\beta}_1$ & 0.9820 & 1.0000 & 1.0000 & 1.0000 & 0.5840 & 1.0000 & 1.0000 & 1.0000 & 1.0000 & 0.5220\\Power $\hat{\beta}_2$ & 0.9660 & 1.0000 & 1.0000 & 1.0000 & 0.5780 & 1.0000 & 1.0000 & 1.0000 & 1.0000 & 0.5460\\ \hline 
\end{tabular} 

\end{table}


\begin{table}[]
\caption{DGP 3, weakly autocorrelated errors}
    \centering
 \begin{tabular}{lccccccccc} 
\hline \multicolumn{8}{c}{MSE of the Time-Varying Individual Effects} \\ \hline 
$n$ & $T$ & KSS & $ \text{Bai}_{\hat{d} = 8}$ & $\text{Bai}_{\hat{d} = d}$ & Eup & $\hat{d}_{KSS}$ & $\hat{d}_{Eup}$ \\
\hline
30 & 12 &  1.3502  &  1.0141  &  0.2257  &  0.4084  &  2.7240  &  2.9340  \\
& 30 &  10.9827  &  0.6467  &  0.1290  &  0.1599  &  0.6020  &  2.2270  \\
100 & 12 &  0.5920  &  0.8306  &  0.1828  &  0.2452  &  2.7040  &  2.3700  \\
& 30 &  9.5471  &  0.4658  &  0.0861  &  0.0872  &  0.3590  &  2.0030  \\
300 & 12 &  0.3459  &  0.7769  &  0.1736  &  0.2041  &  2.6860  &  2.1940  \\
& 30 &  9.0764  &  0.3921  &  0.0733  &  0.0733  &  0.1260  &  2.0000  \\
\end{tabular} 
\begin{tabular}{ccccccccccc} 
\hline 
\multicolumn{10}{c}{MSE, Bias, Variance and Power for the Common Slope Coefficients} \\ \hline 
& \multicolumn{5}{c}{$T=12$} & \multicolumn{5}{c}{$T=30$} \\ \cline{2-6} \cline{7-11} 
& KSS & $ \text{Bai}_{\hat{d} = 8}$ & $\text{Bai}_{\hat{d} = d}$& Eup & Within & KSS & \text{Bai}_{\hat{d} = 8} & \text{Bai}_{\hat{d} = d} & Eup & Within \\\multicolumn{8}{l}{$n = 30 } \\$\text{MSE}_\hat{\beta}$ & 0.0086 & 0.0100 & 0.0018 & 0.0023 & 0.0452 & 0.0166 & 0.0013 & 0.0006 & 0.0006 & 0.0155\\Bias $\hat{\beta}_1$ & -0.0002 & -0.0012 & -0.0001 & -0.0019 & 0.0029 & -0.0007 & -0.0022 & 0.0000 & -0.0002 & -0.0008\\Bias $\hat{\beta}_2$ & 0.0043 & -0.0018 & 0.0000 & 0.0006 & 0.0043 & -0.0011 & 0.0008 & 0.0001 & -0.0002 & -0.0005\\$\text{Var}(\hat{\beta}_1)$ & 0.0926 & 0.0258 & 0.0292 & 0.0281 & 0.2046 & 0.0926 & 0.0170 & 0.0198 & 0.0197 & 0.1181\\$\text{Var}(\hat{\beta}_2)$ & 0.0926 & 0.0259 & 0.0292 & 0.0281 & 0.2042 & 0.0926 & 0.0170 & 0.0198 & 0.0197 & 0.1192\\Power $\hat{\beta}_1$ & 1.0000 & 1.0000 & 1.0000 & 1.0000 & 0.6830 & 0.9820 & 1.0000 & 1.0000 & 1.0000 & 0.9730\\Power $\hat{\beta}_2$ & 1.0000 & 1.0000 & 1.0000 & 1.0000 & 0.7030 & 0.9870 & 1.0000 & 1.0000 & 1.0000 & 0.9760\\ \hline 
\multicolumn{8}{l}{$n = 100 } \\$\text{MSE}_\hat{\beta}$ & 0.0028 & 0.0017 & 0.0005 & 0.0005 & 0.0125 & 0.0044 & 0.0003 & 0.0002 & 0.0002 & 0.0045\\Bias $\hat{\beta}_1$ & 0.0025 & 0.0004 & 0.0000 & 0.0006 & 0.0058 & -0.0004 & -0.0001 & 0.0000 & -0.0001 & -0.0006\\Bias $\hat{\beta}_2$ & -0.0013 & -0.0015 & -0.0001 & -0.0005 & 0.0023 & -0.0004 & -0.0002 & 0.0000 & -0.0001 & -0.0005\\$\text{Var}(\hat{\beta}_1)$ & 0.0490 & 0.0163 & 0.0160 & 0.0160 & 0.1126 & 0.0455 & 0.0101 & 0.0109 & 0.0109 & 0.0649\\$\text{Var}(\hat{\beta}_2)$ & 0.0491 & 0.0163 & 0.0160 & 0.0160 & 0.1123 & 0.0455 & 0.0101 & 0.0109 & 0.0109 & 0.0648\\Power $\hat{\beta}_1$ & 1.0000 & 1.0000 & 1.0000 & 1.0000 & 0.9910 & 1.0000 & 1.0000 & 1.0000 & 1.0000 & 1.0000\\Power $\hat{\beta}_2$ & 1.0000 & 1.0000 & 1.0000 & 1.0000 & 0.9910 & 1.0000 & 1.0000 & 1.0000 & 1.0000 & 1.0000\\ \hline 
\multicolumn{8}{l}{$n = 300 } \\$\text{MSE}_\hat{\beta}$ & 0.0009 & 0.0005 & 0.0002 & 0.0002 & 0.0045 & 0.0010 & 0.0001 & 0.0001 & 0.0001 & 0.0013\\Bias $\hat{\beta}_1$ & -0.0018 & -0.0001 & -0.0001 & -0.0002 & 0.0015 & 0.0025 & 0.0008 & 0.0008 & 0.0008 & 0.0015\\Bias $\hat{\beta}_2$ & -0.0006 & -0.0001 & 0.0003 & 0.0003 & -0.0037 & -0.0012 & -0.0007 & -0.0009 & -0.0009 & -0.0027\\$\text{Var}(\hat{\beta}_1)$ & 0.0279 & 0.0104 & 0.0093 & 0.0093 & 0.0662 & 0.0236 & 0.0061 & 0.0063 & 0.0063 & 0.0382\\$\text{Var}(\hat{\beta}_2)$ & 0.0279 & 0.0104 & 0.0093 & 0.0093 & 0.0662 & 0.0236 & 0.0061 & 0.0063 & 0.0063 & 0.0383\\Power $\hat{\beta}_1$ & 1.0000 & 1.0000 & 1.0000 & 1.0000 & 1.0000 & 1.0000 & 1.0000 & 1.0000 & 1.0000 & 1.0000\\Power $\hat{\beta}_2$ & 1.0000 & 1.0000 & 1.0000 & 1.0000 & 1.0000 & 1.0000 & 1.0000 & 1.0000 & 1.0000 & 1.0000\\ \hline 
\end{tabular} 

\end{table}


\begin{table}[]
\caption{DGP 4,  weakly autocorrelated errors}
    \centering
 \begin{tabular}{lcccccc} 
\hline \multicolumn{6}{c}{MSE of Effects} \\ \hline 
$n$ & $T$ & KSS & Eup & $d_{KSS}$ & $d_{Eup}$ \\
\hline
30 & 12 &  0.5684  &  1.1062  &  1.0750  &  2.8690  \\
& 30 &  0.1546  &  0.2199  &  1.0000  &  1.0060  \\
100 & 12 &  0.2260  &  0.2187  &  1.0020  &  1.0200  \\
& 30 &  0.1782  &  0.1104  &  2.1380  &  1.0000  \\
300 & 12 &  0.1037  &  0.0973  &  1.0000  &  1.0000  \\
& 30 &  0.0436  &  0.0460  &  1.0000  &  1.0000  \\
\end{tabular} 
\begin{tabular}{ccccccc} 
\hline 
\multicolumn{7}{c}{MSE, Bias, Variance for Coefficients} \\ \hline 
& \multicolumn{3}{c}{$T=12$} & \multicolumn{3}{c}{$T=30$} \\ \cline{2-4} \cline{5-7} 
& KSS & Eup & Within & KSS & Eup & Within \\\multicolumn{7}{l}{$n = 30 } \\MSE  & 0.00349 & 0.00531 & 0.00294 & 0.00109 & 0.00117 & 0.00102\\ BIAS1  & -0.00218 & -0.00254 & -0.00011 & -0.00068 & 7e-05 & -1e-05\\ BIAS2  & -0.00165 & -0.00340 & -0.00158 & -0.00028 & 0.00018 & 0.00026\\ VAR1  & 0.06061 & 0.03602 & 0.05239 & 0.03364 & 0.02559 & 0.03176\\ VAR2  & 0.06066 & 0.03595 & 0.05246 & 0.03361 & 0.02554 & 0.03172\\ \hline 
\multicolumn{7}{l}{$n = 100 } \\MSE  & 0.00099 & 0.00084 & 0.00078 & 4e-04 & 0.00037 & 0.00035\\ BIAS1  & 4e-05 & 0.00018 & 0.00045 & -0.00037 & 0.00032 & 0.00052\\ BIAS2  & 0.00097 & 0.00100 & 0.00105 & -0.00014 & -0.00018 & 4e-05\\ VAR1  & 0.03226 & 0.02174 & 0.02864 & 0.01991 & 0.01536 & 0.01698\\ VAR2  & 0.03218 & 0.02169 & 0.02859 & 0.01992 & 0.01535 & 0.01699\\ \hline 
\multicolumn{7}{l}{$n = 300 } \\MSE  & 0.00034 & 0.00027 & 0.00027 & 0.00011 & 0.00011 & 0.00011\\ BIAS1  & 0.00082 & 0.00018 & 0.00033 & 0.00037 & 0.00031 & 0.00034\\ BIAS2  & -0.00153 & -0.00176 & -0.00164 & 0.00037 & 0.00045 & 0.00049\\ VAR1  & 0.01889 & 0.01286 & 0.01705 & 0.01046 & 0.00859 & 0.01004\\ VAR2  & 0.01889 & 0.01283 & 0.01705 & 0.01048 & 0.00859 & 0.01005\\ \hline 
\end{tabular} 

\end{table}


\begin{table}[]
\caption{DGP 5,  weakly autocorrelated errors}
    \centering
 \begin{tabular}{lcccccc} 
\hline \multicolumn{6}{c}{MSE of Effects} \\ \hline 
$n$ & $T$ & KSS & Eup & $d_{KSS}$ & $d_{Eup}$ \\
\hline
30 & 12 &  0.8985  &  0.4430  &  1.5750  &  2.6990  \\
& 30 &  0.3237  &  0.0822  &  1.5040  &  1.0170  \\
100 & 12 &  0.3809  &  0.1380  &  1.4920  &  1.1530  \\
& 30 &  0.2478  &  0.0374  &  0.9460  &  1.0000  \\
300 & 12 &  0.2260  &  0.0850  &  1.1240  &  1.0000  \\
& 30 &  0.1804  &  0.0471  &  2.7760  &  1.0000  \\
\end{tabular} 
\begin{tabular}{ccccccc} 
\hline 
\multicolumn{7}{c}{MSE, Bias, Variance for Coefficients} \\ \hline 
& \multicolumn{3}{c}{$T=12$} & \multicolumn{3}{c}{$T=30$} \\ \cline{2-4} \cline{5-7} 
& KSS & Eup & Within & KSS & Eup & Within \\\multicolumn{7}{l}{$n = 30 } \\MSE  & 0.00519 & 0.00235 & 0.00616 & 0.00156 & 0.00058 & 0.00155\\ BIAS1  & -0.00191 & -0.00077 & -0.00062 & 0.00104 & 0.0012 & 0.00085\\ BIAS2  & -4e-04 & 0.00016 & -0.00237 & -0.00194 & -0.00118 & -0.00146\\ VAR1  & 0.07137 & 0.02817 & 0.06837 & 0.03896 & 0.02021 & 0.03589\\ VAR2  & 0.07148 & 0.02813 & 0.06839 & 0.03885 & 0.02022 & 0.03579\\ \hline 
\multicolumn{7}{l}{$n = 100 } \\MSE  & 0.00156 & 0.00052 & 0.00179 & 4e-04 & 0.00013 & 0.00036\\ BIAS1  & 0.00124 & 5e-04 & 0.00052 & 0.00052 & 0.00019 & 0.00107\\ BIAS2  & 0.00037 & -0.00066 & -0.00056 & -4e-05 & -1e-04 & -0.00022\\ VAR1  & 0.03802 & 0.01654 & 0.03669 & 0.01988 & 0.01040 & 0.01947\\ VAR2  & 0.03800 & 0.01654 & 0.03666 & 0.01988 & 0.01040 & 0.01948\\ \hline 
\multicolumn{7}{l}{$n = 300 } \\MSE  & 0.00056 & 0.00014 & 0.00067 & 0.00015 & 6e-05 & 0.00015\\ BIAS1  & -0.00036 & -0.00029 & -0.00169 & -2e-04 & 0.0000 & 0.00041\\ BIAS2  & 0.00098 & 0.00034 & 0.00067 & -7e-05 & -4e-05 & -0.00028\\ VAR1  & 0.02170 & 0.00940 & 0.02165 & 0.01255 & 0.00670 & 0.01122\\ VAR2  & 0.02171 & 0.00940 & 0.02165 & 0.01255 & 0.00670 & 0.01123\\ \hline 
\end{tabular} 

\end{table}


\subsection(Strongly autocrorrelated error terms}


\begin{table}[]
\caption{DGP 1, strongly autocorrelated errors}
    \centering
 \begin{tabular}{lcccccc} 
\hline \multicolumn{6}{c}{MSE of Effects} \\ \hline 
$n$ & $T$ & KSS & Eup & $d_{KSS}$ & $d_{Eup}$ \\
\hline
30 & 12 &  0.9568  &  1.0161  &  3.5880  &  3.0000  \\
& 30 &  0.8787  &  0.8961  &  7.4240  &  4.9990  \\
100 & 12 &  0.8689  &  0.8379  &  3.8530  &  3.0000  \\
& 30 &  0.7436  &  0.6471  &  7.9210  &  5.0000  \\
300 & 12 &  0.7658  &  0.7169  &  3.7700  &  3.0000  \\
& 30 &  0.8059  &  0.6862  &  8.6120  &  5.0000  \\
\end{tabular} 
\begin{tabular}{ccccccc} 
\hline 
\multicolumn{7}{c}{MSE, Bias, Variance for Coefficients} \\ \hline 
& \multicolumn{3}{c}{$T=12$} & \multicolumn{3}{c}{$T=30$} \\ \cline{2-4} \cline{5-7} 
& KSS & Eup & Within & KSS & Eup & Within \\\multicolumn{7}{l}{$n = 30 } \\MSE  & 0.00127 & 0.00190 & 0.03288 & 0.00038 & 0.00087 & 0.01114\\ BIAS1  & -8e-05 & 0.00154 & 0.00161 & -7e-04 & 0.00029 & -0.00026\\ BIAS2  & -0.00091 & 0.00013 & 0.00225 & -0.00142 & -0.00122 & -0.00021\\ VAR1  & 0.04469 & 0.02678 & 0.12672 & 0.02617 & 0.01832 & 0.07329\\ VAR2  & 0.04479 & 0.02694 & 0.12687 & 0.02619 & 0.01836 & 0.07328\\ \hline 
\multicolumn{7}{l}{$n = 100 } \\MSE  & 0.00033 & 0.00047 & 0.00896 & 2e-04 & 0.00033 & 0.00412\\ BIAS1  & -0.00029 & -0.00014 & -0.00482 & -7e-04 & -0.00087 & 0.00015\\ BIAS2  & 0.00062 & 0.00069 & -0.00147 & 0.00015 & 0.00081 & -0.00150\\ VAR1  & 0.02209 & 0.01441 & 0.06839 & 0.01669 & 0.01269 & 0.04008\\ VAR2  & 0.02202 & 0.01437 & 0.06813 & 0.01670 & 0.01270 & 0.04011\\ \hline 
\multicolumn{7}{l}{$n = 300 } \\MSE  & 0.00017 & 0.00021 & 0.00321 & 5e-05 & 9e-05 & 0.00105\\ BIAS1  & -0.00014 & -0.00097 & 0.00295 & -1e-05 & -0.00037 & -0.00033\\ BIAS2  & 0.00053 & 0.00044 & 0.00043 & -0.00021 & 0.00011 & 0.00240\\ VAR1  & 0.01491 & 0.00984 & 0.04039 & 0.00806 & 0.00658 & 0.02337\\ VAR2  & 0.01493 & 0.00983 & 0.04042 & 0.00806 & 0.00658 & 0.02338\\ \hline 
\end{tabular} 

\end{table}
    
\begin{table}[]
\caption{DGP 2, strongly autocorrelated errors}
    \centering
 \begin{tabular}{lcccccc} 
\hline \multicolumn{6}{c}{MSE of Effects} \\ \hline 
$n$ & $T$ & KSS & Eup & $d_{KSS}$ & $d_{Eup}$ \\
\hline
30 & 12 &  33.2078  &  1.0155  &  1.1730  &  3.0000  \\
& 30 &  17.2788  &  1.0266  &  1.0940  &  5.0000  \\
100 & 12 &  22.3847  &  0.6583  &  1.0390  &  3.0000  \\
& 30 &  12.9281  &  0.6278  &  1.0100  &  5.0000  \\
300 & 12 &  19.9682  &  0.6957  &  1.0000  &  3.0000  \\
& 30 &  11.3203  &  0.6558  &  1.0000  &  5.0000  \\
\end{tabular} 
\begin{tabular}{ccccccc} 
\hline 
\multicolumn{7}{c}{MSE, Bias, Variance for Coefficients} \\ \hline 
& \multicolumn{3}{c}{$T=12$} & \multicolumn{3}{c}{$T=30$} \\ \cline{2-4} \cline{5-7} 
& KSS & Eup & Within & KSS & Eup & Within \\\multicolumn{7}{l}{$n = 30 } \\MSE  & 0.15417 & 0.00211 & 0.85703 & 0.05914 & 0.00161 & 0.75538\\ BIAS1  & 0.01148 & -0.00085 & 0.05412 & -0.01147 & 0.00125 & -0.01639\\ BIAS2  & -0.00751 & 0.00176 & 0.03704 & 0.01053 & -0.00104 & 0.08992\\ VAR1  & 0.28532 & 0.02105 & 0.66218 & 0.16230 & 0.01506 & 0.59791\\ VAR2  & 0.28493 & 0.02108 & 0.66315 & 0.16260 & 0.01503 & 0.59853\\ \hline 
\multicolumn{7}{l}{$n = 100 } \\MSE  & 0.04873 & 0.00061 & 0.24145 & 0.01912 & 0.00026 & 0.28217\\ BIAS1  & -0.00033 & -0.00063 & -0.00145 & -0.01103 & -0.00127 & 0.01301\\ BIAS2  & -0.00227 & 0.00171 & -0.01092 & -0.00134 & -0.00013 & 0.02617\\ VAR1  & 0.15406 & 0.01406 & 0.34746 & 0.08470 & 0.00964 & 0.32634\\ VAR2  & 0.15419 & 0.01406 & 0.34782 & 0.08458 & 0.00963 & 0.32649\\ \hline 
\multicolumn{7}{l}{$n = 300 } \\MSE  & 0.01633 & 0.00015 & 0.09004 & 0.00627 & 6e-05 & 0.10119\\ BIAS1  & 0.00284 & -9e-05 & -0.00115 & -0.00186 & -0.00029 & -0.02471\\ BIAS2  & 0.00673 & -0.00029 & -0.00368 & -0.00196 & -9e-05 & -0.01179\\ VAR1  & 0.08701 & 0.00760 & 0.20583 & 0.04624 & 0.00532 & 0.19672\\ VAR2  & 0.08693 & 0.00759 & 0.20552 & 0.04624 & 0.00532 & 0.19669\\ \hline 
\end{tabular} 

\end{table}


\begin{table}[]
\caption{DGP 3, strongly autocorrelated errors}
    \centering
 \begin{tabular}{lccccccccc} 
\hline \multicolumn{8}{c}{MSE of the Time-Varying Individual Effects} \\ \hline 
$n$ & $T$ & KSS & $ \text{Bai}_{\hat{d} = 8}$ & $\text{Bai}_{\hat{d} = d}$ & Eup & $\hat{d}_{KSS}$ & $\hat{d}_{Eup}$ \\
\hline
30 & 12 &  1.1998  &  1.1498  &  0.2638  &  0.8048  &  4.8090  &  3.0000  \\
& 30 &  11.1301  &  0.9433  &  0.1405  &  0.7252  &  0.7570  &  5.0000  \\
100 & 12 &  0.9696  &  0.9996  &  0.2191  &  0.6374  &  5.5880  &  3.0000  \\
& 30 &  9.7117  &  0.7990  &  0.0918  &  0.5846  &  0.5320  &  5.0000  \\
300 & 12 &  0.9124  &  0.9500  &  0.2044  &  0.5839  &  6.1740  &  3.0000  \\
& 30 &  9.2486  &  0.7545  &  0.0785  &  0.5432  &  0.3480  &  5.0000  \\
\end{tabular} 
\begin{tabular}{ccccccccccc} 
\hline 
\multicolumn{10}{c}{MSE, Bias, Variance and Power for the Common Slope Coefficients} \\ \hline 
& \multicolumn{5}{c}{$T=12$} & \multicolumn{5}{c}{$T=30$} \\ \cline{2-6} \cline{7-11} 
& KSS & $ \text{Bai}_{\hat{d} = 8}$ & $\text{Bai}_{\hat{d} = d}$& Eup & Within & KSS & \text{Bai}_{\hat{d} = 8} & \text{Bai}_{\hat{d} = d} & Eup & Within \\\multicolumn{8}{l}{$n = 30 } \\$\text{MSE}_\hat{\beta}$ & 0.0026 & 0.0040 & 0.0030 & 0.0030 & 0.0433 & 0.0172 & 0.0013 & 0.0010 & 0.0011 & 0.0155\\Bias $\hat{\beta}_1$ & 0.0017 & -0.0003 & 0.0034 & 0.0011 & 0.0042 & -0.0026 & -0.0033 & -0.0021 & -0.0036 & -0.0011\\Bias $\hat{\beta}_2$ & -0.0013 & -0.0056 & -0.0036 & -0.0016 & -0.0066 & -0.0064 & -0.0005 & 0.0022 & -0.0011 & -0.0054\\$\text{Var}(\hat{\beta}_1)$ & 0.0562 & 0.0208 & 0.0306 & 0.0262 & 0.2039 & 0.0967 & 0.0146 & 0.0236 & 0.0172 & 0.1212\\$\text{Var}(\hat{\beta}_2)$ & 0.0563 & 0.0209 & 0.0306 & 0.0262 & 0.2059 & 0.0969 & 0.0145 & 0.0236 & 0.0172 & 0.1205\\Power $\hat{\beta}_1$ & 1.0000 & 1.0000 & 1.0000 & 1.0000 & 0.6890 & 0.9800 & 1.0000 & 1.0000 & 1.0000 & 0.9630\\Power $\hat{\beta}_2$ & 1.0000 & 1.0000 & 1.0000 & 1.0000 & 0.6600 & 0.9890 & 1.0000 & 1.0000 & 1.0000 & 0.9750\\ \hline 
\multicolumn{8}{l}{$n = 100 } \\$\text{MSE}_\hat{\beta}$ & 0.0007 & 0.0008 & 0.0008 & 0.0008 & 0.0130 & 0.0044 & 0.0003 & 0.0003 & 0.0003 & 0.0040\\Bias $\hat{\beta}_1$ & -0.0001 & -0.0014 & -0.0001 & 0.0001 & -0.0012 & -0.0021 & -0.0003 & 0.0007 & 0.0007 & -0.0018\\Bias $\hat{\beta}_2$ & 0.0004 & 0.0015 & 0.0003 & -0.0009 & 0.0007 & 0.0017 & -0.0009 & -0.0006 & -0.0007 & 0.0015\\$\text{Var}(\hat{\beta}_1)$ & 0.0277 & 0.0144 & 0.0170 & 0.0151 & 0.1127 & 0.0487 & 0.0087 & 0.0130 & 0.0099 & 0.0660\\$\text{Var}(\hat{\beta}_2)$ & 0.0277 & 0.0143 & 0.0170 & 0.0151 & 0.1122 & 0.0487 & 0.0087 & 0.0130 & 0.0099 & 0.0663\\Power $\hat{\beta}_1$ & 1.0000 & 1.0000 & 1.0000 & 1.0000 & 0.9880 & 1.0000 & 1.0000 & 1.0000 & 1.0000 & 1.0000\\Power $\hat{\beta}_2$ & 1.0000 & 1.0000 & 1.0000 & 1.0000 & 0.9910 & 1.0000 & 1.0000 & 1.0000 & 1.0000 & 1.0000\\ \hline 
\multicolumn{8}{l}{$n = 300 } \\$\text{MSE}_\hat{\beta}$ & 0.0003 & 0.0002 & 0.0003 & 0.0003 & 0.0048 & 0.0015 & 0.0001 & 0.0001 & 0.0001 & 0.0016\\Bias $\hat{\beta}_1$ & 0.0006 & 0.0006 & -0.0006 & -0.0015 & -0.0012 & 0.0030 & 0.0002 & 0.0002 & -0.0001 & 0.0030\\Bias $\hat{\beta}_2$ & -0.0011 & -0.0016 & 0.0005 & 0.0006 & 0.0041 & -0.0022 & -0.0002 & -0.0005 & 0.0004 & -0.0030\\$\text{Var}(\hat{\beta}_1)$ & 0.0154 & 0.0094 & 0.0099 & 0.0090 & 0.0665 & 0.0260 & 0.0054 & 0.0075 & 0.0059 & 0.0388\\$\text{Var}(\hat{\beta}_2)$ & 0.0154 & 0.0094 & 0.0099 & 0.0090 & 0.0663 & 0.0261 & 0.0054 & 0.0075 & 0.0059 & 0.0390\\Power $\hat{\beta}_1$ & 1.0000 & 1.0000 & 1.0000 & 1.0000 & 1.0000 & 1.0000 & 1.0000 & 1.0000 & 1.0000 & 1.0000\\Power $\hat{\beta}_2$ & 1.0000 & 1.0000 & 1.0000 & 1.0000 & 1.0000 & 1.0000 & 1.0000 & 1.0000 & 1.0000 & 1.0000\\ \hline 
\end{tabular} 

\end{table}


\begin{table}[]
\caption{DGP 4,  strongly autocorrelated errors}
    \centering
 \begin{tabular}{lccccccccc} 
\hline \multicolumn{8}{c}{MSE of the Time-Varying Individual Effects} \\ \hline 
$n$ & $T$ & KSS & $ \text{Bai}_{\hat{d} = 8}$ & $\text{Bai}_{\hat{d} = d}$ & Eup & $\hat{d}_{KSS}$ & $\hat{d}_{Eup}$ \\
\hline
30 & 12 &  0.9663  &  1.2053  &  0.8927  &  1.0826  &  3.3060  &  3.0000  \\
& 30 &  0.8168  &  0.9879  &  0.4660  &  0.9004  &  6.9860  &  4.9890  \\
100 & 12 &  0.8082  &  0.9983  &  0.4887  &  0.8033  &  3.5180  &  3.0000  \\
& 30 &  0.7663  &  0.8055  &  0.2506  &  0.6807  &  7.9990  &  4.9990  \\
300 & 12 &  0.7758  &  0.9536  &  0.4093  &  0.7453  &  3.6640  &  3.0000  \\
& 30 &  0.7558  &  0.7683  &  0.1992  &  0.6368  &  8.3240  &  5.0000  \\
\end{tabular} 
\begin{tabular}{ccccccccccc} 
\hline 
\multicolumn{10}{c}{MSE, Bias, Variance and Power for the Common Slope Coefficients} \\ \hline 
& \multicolumn{5}{c}{$T=12$} & \multicolumn{5}{c}{$T=30$} \\ \cline{2-6} \cline{7-11} 
& KSS & $ \text{Bai}_{\hat{d} = 8}$ & $\text{Bai}_{\hat{d} = d}$& Eup & Within & KSS & \text{Bai}_{\hat{d} = 8} & \text{Bai}_{\hat{d} = d} & Eup & Within \\\multicolumn{8}{l}{$n = 30 } \\$\text{MSE}_\hat{\beta}$ & 0.0017 & 0.0043 & 0.0037 & 0.0026 & 0.0032 & 0.0006 & 0.0013 & 0.0017 & 0.0013 & 0.0015\\Bias $\hat{\beta}_1$ & -0.0026 & 0.0030 & -0.0021 & -0.0037 & -0.0021 & 0.0002 & 0.0005 & -0.0010 & -0.0010 & -0.0004\\Bias $\hat{\beta}_2$ & -0.0001 & -0.0049 & 0.0005 & -0.0005 & 0.0007 & -0.0004 & 0.0004 & -0.0010 & -0.0003 & -0.0010\\$\text{Var}(\hat{\beta}_1)$ & 0.0483 & 0.0250 & 0.0344 & 0.0270 & 0.0538 & 0.0310 & 0.0177 & 0.0280 & 0.0199 & 0.0375\\$\text{Var}(\hat{\beta}_2)$ & 0.0483 & 0.0247 & 0.0345 & 0.0271 & 0.0540 & 0.0310 & 0.0177 & 0.0280 & 0.0200 & 0.0376\\Power $\hat{\beta}_1$ & 1.0000 & 1.0000 & 1.0000 & 1.0000 & 1.0000 & 1.0000 & 1.0000 & 1.0000 & 1.0000 & 1.0000\\Power $\hat{\beta}_2$ & 1.0000 & 1.0000 & 1.0000 & 1.0000 & 1.0000 & 1.0000 & 1.0000 & 1.0000 & 1.0000 & 1.0000\\ \hline 
\multicolumn{8}{l}{$n = 100 } \\$\text{MSE}_\hat{\beta}$ & 0.0005 & 0.0008 & 0.0010 & 0.0007 & 0.0009 & 0.0002 & 0.0003 & 0.0005 & 0.0003 & 0.0004\\Bias $\hat{\beta}_1$ & -0.0006 & -0.0005 & -0.0004 & 0.0002 & -0.0004 & -0.0001 & 0.0000 & 0.0008 & -0.0003 & 0.0007\\Bias $\hat{\beta}_2$ & -0.0006 & -0.0011 & -0.0024 & -0.0008 & -0.0021 & -0.0002 & -0.0002 & -0.0007 & -0.0007 & -0.0008\\$\text{Var}(\hat{\beta}_1)$ & 0.0250 & 0.0167 & 0.0194 & 0.0155 & 0.0296 & 0.0157 & 0.0109 & 0.0159 & 0.0118 & 0.0207\\$\text{Var}(\hat{\beta}_2)$ & 0.0250 & 0.0167 & 0.0193 & 0.0155 & 0.0296 & 0.0157 & 0.0109 & 0.0160 & 0.0118 & 0.0210\\Power $\hat{\beta}_1$ & 1.0000 & 1.0000 & 1.0000 & 1.0000 & 1.0000 & 1.0000 & 1.0000 & 1.0000 & 1.0000 & 1.0000\\Power $\hat{\beta}_2$ & 1.0000 & 1.0000 & 1.0000 & 1.0000 & 1.0000 & 1.0000 & 1.0000 & 1.0000 & 1.0000 & 1.0000\\ \hline 
\multicolumn{8}{l}{$n = 300 } \\$\text{MSE}_\hat{\beta}$ & 0.0002 & 0.0002 & 0.0003 & 0.0002 & 0.0003 & 0.0001 & 0.0001 & 0.0002 & 0.0001 & 0.0002\\Bias $\hat{\beta}_1$ & 0.0004 & 0.0010 & 0.0012 & 0.0007 & 0.0011 & -0.0002 & -0.0004 & 0.0002 & 0.0000 & 0.0001\\Bias $\hat{\beta}_2$ & -0.0004 & 0.0002 & 0.0002 & -0.0001 & 0.0002 & 0.0003 & 0.0002 & 0.0000 & 0.0004 & -0.0001\\$\text{Var}(\hat{\beta}_1)$ & 0.0143 & 0.0105 & 0.0115 & 0.0092 & 0.0173 & 0.0088 & 0.0065 & 0.0094 & 0.0069 & 0.0120\\$\text{Var}(\hat{\beta}_2)$ & 0.0143 & 0.0105 & 0.0115 & 0.0092 & 0.0174 & 0.0088 & 0.0065 & 0.0094 & 0.0069 & 0.0121\\Power $\hat{\beta}_1$ & 1.0000 & 1.0000 & 1.0000 & 1.0000 & 1.0000 & 1.0000 & 1.0000 & 1.0000 & 1.0000 & 1.0000\\Power $\hat{\beta}_2$ & 1.0000 & 1.0000 & 1.0000 & 1.0000 & 1.0000 & 1.0000 & 1.0000 & 1.0000 & 1.0000 & 1.0000\\ \hline 
\end{tabular} 

\end{table}


\begin{table}[]
\caption{DGP 5,  strongly autocorrelated errors}
    \centering
 \begin{tabular}{lcccccc} 
\hline \multicolumn{6}{c}{MSE of Effects} \\ \hline 
$n$ & $T$ & KSS & Eup & $d_{KSS}$ & $d_{Eup}$ \\
\hline
30 & 12 &  0.9917  &  0.8498  &  3.3010  &  3.0000  \\
& 30 &  0.8612  &  0.8082  &  7.4440  &  5.0000  \\
100 & 12 &  0.8570  &  0.7783  &  3.8670  &  3.0000  \\
& 30 &  0.7009  &  0.5687  &  7.7990  &  5.0000  \\
300 & 12 &  0.8019  &  0.7132  &  3.9140  &  3.0000  \\
& 30 &  0.7805  &  0.6169  &  8.7840  &  5.0000  \\
\end{tabular} 
\begin{tabular}{ccccccc} 
\hline 
\multicolumn{7}{c}{MSE, Bias, Variance for Coefficients} \\ \hline 
& \multicolumn{3}{c}{$T=12$} & \multicolumn{3}{c}{$T=30$} \\ \cline{2-4} \cline{5-7} 
& KSS & Eup & Within & KSS & Eup & Within \\\multicolumn{7}{l}{$n = 30 } \\MSE  & 0.00249 & 0.00235 & 0.00618 & 6e-04 & 0.00105 & 0.00189\\ BIAS1  & -8e-05 & 0.0023 & 0.0031 & -0.00016 & 0.00047 & -5e-04\\ BIAS2  & -0.00033 & -0.00011 & -0.00037 & -0.00023 & -0.00173 & -0.00088\\ VAR1  & 0.06017 & 0.02506 & 0.06340 & 0.03120 & 0.01598 & 0.03342\\ VAR2  & 0.06016 & 0.02503 & 0.06359 & 0.03126 & 0.01602 & 0.03343\\ \hline 
\multicolumn{7}{l}{$n = 100 } \\MSE  & 0.00047 & 0.00048 & 0.00181 & 0.00026 & 0.00033 & 0.00058\\ BIAS1  & 0.00122 & 0.00248 & 0.00238 & 0.00039 & -0.00018 & 0.00049\\ BIAS2  & -0.00061 & -0.00092 & -0.00077 & 0.00100 & 6e-04 & 0.00232\\ VAR1  & 0.02623 & 0.01256 & 0.03304 & 0.01870 & 0.01027 & 0.01857\\ VAR2  & 0.02622 & 0.01259 & 0.03305 & 0.01871 & 0.01028 & 0.01858\\ \hline 
\multicolumn{7}{l}{$n = 300 } \\MSE  & 0.00018 & 0.00016 & 6e-04 & 6e-05 & 9e-05 & 0.00021\\ BIAS1  & 0.00052 & 6e-05 & -0.00047 & -8e-05 & -0.00039 & -7e-04\\ BIAS2  & 0.00022 & 0.00045 & 0.00121 & -9e-05 & -0.00026 & 2e-05\\ VAR1  & 0.01576 & 0.00773 & 0.01970 & 0.00912 & 0.00582 & 0.01063\\ VAR2  & 0.01579 & 0.00774 & 0.01970 & 0.00912 & 0.00583 & 0.01063\\ \hline 
\end{tabular} 

\end{table}


\section{Application}

\documentclass{article}
\usepackage{booktabs}

\begin{document}

\begin{table}[ht]
\centering
\caption{Testing the Presence of Interactive Effects - Test of Kneip, Sickles, and Song (2012)}
\label{tab:interactive_effects}
\begin{tabular}{lcccc}
\toprule
Test-Statistic & p-value & crit.-value & sig.-level \\
\midrule
38.49 & 0.00 & 2.33 & 0.01 \\
\bottomrule
\end{tabular}
\end{table}

\begin{table}[ht]
\centering
\caption{Slope-Coefficients}
\label{tab:slope_coefficients}
\begin{tabular}{lcccccc}
\toprule
 & Estimate & Std.Err & Z value & Pr(>z) \\
\midrule
dem & -0.000210 & 0.000259 & -0.81 & 0.418 \\
lag(l.d.gdp.a, 1) & 0.268000 & 0.022200 & 12.00 & $<$2e-16 *** \\
lag(l.d.gdp.a, 2) & 0.021100 & 0.022700 & 0.93 & 0.352 \\
lag(l.d.gdp.a, 3) & -0.019400 & 0.022200 & -0.87 & 0.384 \\
lag(l.d.gdp.a, 4) & 0.021600 & 0.020800 & 1.04 & 0.300 \\
\bottomrule
\end{tabular}
\end{table}

\textbf{Call:}

\texttt{Eup.default(formula = l.d.gdp.a \textasciitilde\ dem + lag(l.d.gdp.a, 1) + lag(l.d.gdp.a, 2) + lag(l.d.gdp.a, 3) + lag(l.d.gdp.a, 4) - 1,}\\
\texttt{additive.effects = "twoways", dim.criterion = "PC3", error.type = 5)}

\textbf{Residuals:}

\begin{tabular}{lllll}
Min & 1Q & Median & 3Q & Max \\
-0.025100 & -0.001840 & 0.000138 & 0.002140 & 0.019000 \\
\end{tabular}

\textbf{Additive Effects Type:} twoways

\textbf{Dimension of the Unobserved Factors:} 7

\textbf{Residual standard error:} 0.004821 on 2325 degrees of freedom, \textbf{R-squared:} 0.7418

\end{document}



\begin{table}[ht]
\centering
\caption{Bootstrap results} 
\begin{tabular}{rrr}
  \hline 
 & lower\_bound\_95 & upper\_bound\_95 \\ 
  \hline 
dem & -0.61 & -0.24 \\ 
  inv & 1.17 & 3.32 \\ 
  l.capital & -0.27 & 0.28 \\ 
  tfp & -0.28 & 0.44 \\ 
  gvt & -7.11 & -0.99 \\ 
  gvt2 & -8.48 & 5.04 \\ 
  hc & 0.84 & 1.23 \\ 
  lifeexp & 0.07 & 0.09 \\ 
  trade & -2.90 & -2.02 \\ 
   \hline 
\end{tabular}
\end{table}


\bibliography{bibliograpy.bib}


\end{document} 
